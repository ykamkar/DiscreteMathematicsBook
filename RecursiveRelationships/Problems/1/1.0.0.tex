\p
الف)\begin{center}
			    
			$ A(x) = 1 + x + x^2 + ... = \frac{1}{1 - x} $
			\medbreak
			$ A(5x) = 1 + 5x + 25x^2 + ... = \frac{1}{1 - 5x} $
			\end{center} 
			
			ب) \begin{center}
			$ A(x) = 1 + x + x^2 + ... = \frac{1}{1 - x} $ \medbreak
			$ A(x^2) = 1 + x^2 + x^4 + ... = \frac{1}{1 - x^2} $ \medbreak
			$ 2A(x^2) = 2 + 2x^2 + 2x^4 + ... = \frac{2}{1 - x^2}$ \medbreak
			\end{center}
			
			ج) \begin{center}
	        $ A(x) = 1 + x + x^2 + ... = \frac{1}{1 - x} $ \medbreak
			$ A^2(x) = 1 + 2x + 3x^2 + ... = \frac{1}{(1 - x)^2}$ \medbreak
			$ A^2(x) - 1 = 2x + 3x^2 + 4x^3 + ... = \frac{1}{(1 - x)^2} - 1$\medbreak
			$ \frac{A^2(x) - 1}{x} = 2 + 3x + 4x^2 + ... = \frac{\frac{1}{(1 - x)^2} - 1}{x}$
			\end{center} 
		    برای تولید دنباله 
		    $<1, 2, 3, ...>$
		    می‌توانستیم به جای این که $A(x)$ را در خودش ضرب کنیم، از آن مشتق هم بگیریم.