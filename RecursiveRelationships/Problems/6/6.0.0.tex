	\p
معادله‌ی مشخصه را تشکیل می‌دهیم:
$$r^2 = r + 1$$
$$r_1 = \frac{1 + \sqrt{5}}{2}$$
$$r_2 = \frac{1 - \sqrt{5}}{2}$$
بنابراین داریم:
$$L_n = c_0(\frac{1 + \sqrt{5}}{2})^n + c_1(\frac{1 - \sqrt{5}}{2})^n$$
ضرایب 
$c_0$
و 
$c_1$
را با جای‌گذاری دو جمله‌ی اول و حل دستگاه زیر به‌دست می‌آوریم:
$$L_0 = c_0 + c_1 = 2$$
$$L_1 = c_0\frac{1 + \sqrt{5}}{2} + c_1\frac{1 - \sqrt{5}}{2} = 1$$
$$c_0 = 1, c_1 = 1$$
بنابراین فرمول صریح این رابطه‌ی بازگشتی برابر است با:
$$L_n = (\frac{1 + \sqrt{5}}{2})^n + (\frac{1 - \sqrt{5}}{2})^n$$