\begin{problem}
    چند پلاک خودرو فارسی با شرایط زیر وجود دارد؟

    \begin{enumerate}
        \item 
        هر پلاک شامل دو رقم عدد صحیح، یک حرف از حروف الفبا و پس از آن سه رقم عدد صحیح باشد.
        \problemsolution{
            با توجه به شروط گفته شده برای پلاک، ۵ کاراکتر داریم. دو کاراکتر اول عدد هستند، بنابراین برای هر کدام از آن‌ها ۱۰ حالت وجود دارد. کاراکتر سوم حرف است، بنابراین ۳۲ حالت برای آن نیز موجود است. به ازای هر یک از سه کاراکتر بعدی که عدد هستند نیز ۱۰ حالت داریم. با توجه به این که حالت‌های ممکن برای هر کاراکتر مستقل از سایر کاراکتر‌هاست، از اصل ضرب برای به دست آوردن تعداد کل حالت‌ها استفاده می‌کنیم که برابر می‌شود با:
            $$10^2 \times 32 \times 10^3$$ 
        }

        \item 
        پلاک‌ها مانند قسمت اول هستند با این تفاوت که ارقام تکراری در پلاک‌ها ظاهر نمی‌شوند.
        \problemsolution{
            مانند قسمت قبل، با ضرب کردن تعداد حالت‌های کاراکتر‌ها و طبق اصل ضرب، تعداد پلاک‌های ممکن را به دست می‌آید، اما باید توجه شود که به ازای هر رقمی که در گام اول انتخاب شود، ۹ حالت دیگر برای گام دوم وجود دارد
            (تمام ارقام بجز رقم انتخاب شده در گام اول).
            همچنین این موضوع برای گام‌های بعدی نیز صادق است پس جواب نهایی برابر می‌شود با:
            $$10 \times 9 \times 32 \times 8 \times 7 \times 6$$
        }   
        
        \item 
        پلاک‌ها مانند قسمت دوم هستند با این تفاوت که حرف فارسی می‌تواند در هر جای پلاک
        (کاراکتر اول، دوم الی آخر)
        ظاهر شود.
        \problemsolution{
            می‌توان مسئله را به ۷ گام تقسیم کرد:
            ۱- انتخاب جایگاه حرف،
            ۲- انتخاب حرف،
            ۳ تا ۷ - انتخاب ارقام.
            بنابراین، طبق اصل ضرب، تعداد پلاک‌های ممکن برابر است با:
            $$6 \times 32 \times 10 \times 9 \times 8 \times 7 \times 6$$    
            که ۶ تعداد حالت‌های گام اول است.
        }
    \end{enumerate}
\end{problem}