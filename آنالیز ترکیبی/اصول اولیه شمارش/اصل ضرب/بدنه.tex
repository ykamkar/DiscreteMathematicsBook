\SUBSECTION{اصل ضرب}
\begin{definition}
    \focused{اصل ضرب:}
    اگر بتوان فرایندی را به دو قسمت متوالی تقسیم کرد و
    $n$
    حالت برای انجام قسمت اول و به ازای هر یک از این حالات،
    $m$
    حالت برای انجام قسمت دوم وجود داشته باشد،
    آنگاه
    $n \times m$
    حالت برای انجام شدن فرایند وجود دارد.
\end{definition}
    
\begin{fact}    
    \focused{تعمیم اصل ضرب:}
    اگر بتوان فرایندی را به
    $k$
    قسمت متوالی تقسیم کرد و
    به ازای هر دنباله‌ای از حالت‌های انجام قسمت‌های 
    $1$ تا $i-1$ام،
    $n_i$
    حالت برای اجام قسمت
    $i$ام
    وجود داشته باشد،
    آنگاه
    $\prod\limits_{i=1}^n n_i$
    حالت برای انجام شدن فرایند وجود دارد.
\end{fact}

\NOTE{
    به فرضیات اصل ضرب تعمیم‌یافته توجه کنید:
    \begin{enumerate}
        \item 
        نه فقط مرحله‌ی
        $i-1$ام
        بلکه کل دنباله‌ی اعمال پیشین
        تأثیرگذار است به نحوی که اگر عملی در ابتدای دنباله،
        باعث کم یا زیاد شدن تعداد حالات پیشروی در انتهای دنباله شود،
        قادر به استفاده از این اصل نیستیم.

        \item 
        نیازی نیست حالات انجام کار در مرحله‌ی
        $i$ام
        به ازای تمام دنباله‌های اعمال 
        $1$
        تا
        $i-1$ام
        یکسان باشد، بلکه تنها کافیست تعداد این حالات برابر باشد.
    \end{enumerate}
}

\subfile{مثال ۱/بدنه.tex}
\subfile{مثال ۲/بدنه.tex}
\subfile{مثال ۳/بدنه.tex}
\subfile{مثال ۴/بدنه.tex}