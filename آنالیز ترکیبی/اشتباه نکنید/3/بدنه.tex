\DMMproblem

۶۰ دانشجو در کلاس ریاضیات گسسته حضور دارند.در میان هر ۱۰ نفر از این کلاس , حداقل ۳ نفر نمره مبانی یکسانی دارند. ثابت کنید در این کلاس ۱۵ نفر وجود دارند که نمره مبانی آن‌ها یکسان است.

\DMMproblemWsolution{
    در نظر می‌گیریم حداکثر تعداد تکرار از یک نمره ۱۴ عدد است که در این صورت حداقل به ۵ نمره متفاوت نیاز است . 
    در این صورت باز می‌توان گروه ۱۰ تایی را از دانش آموزان انتخاب کرد که حداکثر دو نفر نمره یکسان داشته باشند. پس فرض اولیه غلط بوده و مشخص می‌شود که لااقل از یکی از نمرات وجود دارد که ۱۵ دانش آموز یا بیشتر 
    آن نمره را دارند.
}

\NOTE{در پاسخ از برهان خلف استفاده شده ولی از آن نام برده نشده است و باید توجه کنیم فرض خلف را حتما بیان کنیم.}
\NOTE{بهتر بود اصل لانه کبوتری که از آن استفاده کرده است را نام می‌برد و نحوه استفاده از آن را مشخص می‌کرد.}
\NOTE{پاسخ کامل نیست. هر اثباتی باید تا گام آخر طی شود. هیچ بخشی از اثبات نباید بدیهی شمرده شود.}

\DMMproblemsolution{
    از برهان خلف استفاده می‌کنیم. فرض خلف : فرض کنید در این کلاس هیچ ۱۵ نفری وجود نداشته باشند که نمره‌ی مبانی آن‌ها یکسان باشد.
    در این صورت حداکثر ۱۴ نفر وجود دارند که نمره‌ی یکسان داشته باشند. بنابراین  طبق اصل لانه کبوتری حداقل به اندازه‌ی سقف
    $\frac{60}{14}$
    یعنی ۵ نمره‌ی متفاوت در کلاس وجود دارد. مسئله را به دو حالت تقسیم می‌کنیم ؛
    \begin{enumerate}
        \item 
        اگر پنج نمره‌ی متمایز وجود داشته باشند که از هر کدام ۲ عضو
        (دو نفر در کلاس که آن نمره را دارند)
        وجود داشته باشد؛ در این صورت از هر کدام از این نمرات دو عضو را درنظر گرفته و به مجموعه‌ای ۱۰ عضوی
        می‌رسیم که هیچ سه نفری در آن نمره‌ی یکسان ندارند که این خلاف فرض مسئله است و به تناقض رسیدیم.
        پس فرض خلف رد شده و حداقل ۱۵ نفر وجود دارند که نمره‌ی یکسانی داشته باشند.
        
        \item 
        اگر پنج نمره‌ی متمایز، هرکدام دارای حداقل دو عضو وجود نداشته باشند؛
        در این صورت 
        $k \le 4$
        نمره‌ی متمایز با بیش از یک عضو داریم
        (مجموعه‌ی این نمرات را A بنامیم)
        که با توجه به فرض خلف، حداکثر تعداد
        $14 \times k$
        عضو را پوشش می‌دهند. بنابراین حداقل 
        $60 - 14k$
        نفر باقی مانده که هیچ دو تایی نمی‌توانند دارای نمره‌ی یکسان باشند
        (در غیر این صورت تعداد نمره‌های متمایز دارای بیشتر مساوی ۲ عضو به حداقل k+1 می‌رسد).
        بنابراین هر یک از این اعضا دارای نمره‌ای متمایز است
        (مجموعه‌ی این اعضا را B بنامیم).
        می‌توان با انتخاب دو عضو از هر نمره‌ی مجموعه‌ی
        A
        و تمام اعضای مجموعه‌‌ی
        B
        به مجموعه‌ای متشکل از
        $N = 2k + 60 - 14k = 60 - 12k$
        عضو رسید که
        $k \le 4 \Rightarrow N \ge 12$
        و هیچ سه عضوی در آن دارای نمره‌ی یکسان نیستند.
        هر ده عضوی از این مجموعه انتخاب شود، نقض فرض مسئله است و به تناقض رسیدیم.
        پس فرض خلف رد شده و حداقل ۱۵ نفر وجود دارند که نمره‌ی یکسانی داشته باشند.
    \end{enumerate}
}