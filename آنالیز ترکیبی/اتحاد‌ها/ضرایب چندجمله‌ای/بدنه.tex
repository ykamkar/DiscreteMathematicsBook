\SUBSECTION{ضرایب چندجمله‌ای}

\begin{fact}
  \focused{ضرایب دوجمله‌ای:}
    اگر
    $n$
    یک عدد صحیح نامنفی باشد، داریم:

    \begin{center}
    $(x+y)^{n} = \sum\limits_{j=0}^n \binom{n}{j} x^{n-j}y^j$
    \end{center}
\end{fact}

\begin{problem}[اثبات قضیه ضرایب دوجمله‌ای]
  قضیه ضرایب دوجمله‌ای را اثبات کنید.

  \problemsolution{
    برای خوانایی بهتر، می‌نویسیم:
    \begin{center}
      $(x+y)^{n} = \prod\limits_{i=1}^n (x+y)_i$
    \end{center}
    که در آن، اندیس‌ها هیچ معنایی نداشته و فقط برای آسان‌تر کردن اشاره به عامل‌های مختلف
    عبارت اضافه شده‌اند.
    حاصل عبارت بالا برابر با حاصل جمع تعدادی جمله می‌باشد که هر جمله از ضرب یک
    $x$ یا $y$
    از عامل
    $i$ام
    ($(x+y)_i$)
    به ازای 
    $1 \leq i \leq n$
    است. بنابراین جمله‌های به شکل
    $x^{n-j}y^j$
    زمانی مشاهده می‌شوند که از
    $j$
    تعداد عوامل، عبارت
    $y$
    و از 
    $n - j$
    تعداد باقی‌مانده، عبارت
    $x$
    انتخاب شود. مشخص است که تعداد تکرار این جمله در حاصل عبارت
    برابر است با تعداد حالات مختلف انتخاب
    $j$
    عامل از عوامل برای آنکه از آن‌ها عبارت
    $y$
    را در جمله مذکور ضرب کنیم. بنابراین تعداد تکرار جمله 
    $x^{n-j}y^j$
    که ضریب این عبارت در خروجی است، برابر است با:
    \begin{center}
      $\binom{n}{j}$
    \end{center}
    با لحاظ کردن نتیجه به دست آمده، برای تمام جمله‌های ممکن، داریم:

    \begin{center}
      $(x+y)^{n} = \sum\limits_{j=0}^n \binom{n}{j} x^{n-j}y^j$
    \end{center}
  }
\end{problem}

\begin{fact}
  برای
  \focused{ضرایب چندجمله‌ای}
  نیز داریم:
  \begin{center}
    $(\sum\limits_{i=1}^k x_i)^n = \sum\limits_{\sum\limits_{i=1}^k j_i = n} (\binom{n}{j_1,j_2,...,j_k} \prod\limits_{i=1}^{k} x_i^{j_i})$
  \end{center}
\end{fact}

نحوه اثبات قضیه ضرایب چندجمله‌ای نیز مشابه نحوه اثبات قضیه ضرایب دوجمله‌ای است.

\NOTE{اگر جملات حاصل از بسط دوجمله‌ای را بر مبنای توان یکی از جملات مرتب کنیم، آنگاه ضرایب دوجمله‌ای یک عبارت درجه $k$
متناظر با مقادیر سطر
$k$ام (از صفر)
مثلث پاسکال خواهند بود.}

\subfile{ضرایب دوجمله‌ای با توان گویا.tex}
\subfile{مثال ۱/بدنه.tex}
\subfile{مثال ۲/بدنه.tex}
\subfile{مثال ۳/بدنه.tex}
\subfile{مثال ۴/بدنه.tex}
