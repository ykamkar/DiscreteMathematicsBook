\SUBSECTION{دوگانه شماری}

دوگانه شماری روشی ترکیبیاتی برای اثبات برابری دو عبارت می‌باشد.
در این روش، مجموعه‌ای و مسئله‌ای از جنس شمارش بر روی آن تعریف می‌شود.
سپس با حل مسئله از دو روش متفاوت، یک‌بار به یک طرف تساوی و بار دیگر
به طرف دیگر تساوی می‌رسیم.
با توجه به یکسان بودن مسئله و یکتا بودن پاسخ مسئله‌ی شمارش،
می‌توان نتیجه گرفت که دو عبارت به دست آمده برابراند.

\begin{problem}
    فرض کنید
    $n,r$
    اعداد صحیح نامنفی باشند و
    $r \leq n$.
    اتحاد زیر را ثابت کنید:
    \begin{center}
        $\binom{n+1}{r+1} = \sum\limits_{j=r}^n \binom{j}{r}$
    \end{center}
      
    \problemsolution{
        از دوگانه شماری استفاده می‌کنیم. می‌خواهیم یگ رشته باینری به طول
        $n+1$
        حاوی
        $r+1$
        بیت
        $1$
        بسازیم. تعداد حالات ساختن این رشته را به دو طریق می‌توان شمرد :
        \begin{enumerate}
        \item 
        تعداد
        $r+1$
        بیت را از بین
        $n+1$
        بیت موجود انتخاب کرده و با
        $1$
        و مابقی بیت‌ها را با
        $0$
        مقداردهی می‌کنیم.
        \begin{center}
            $\binom{n+1}{r+1}$
        \end{center}

        \item 
        ابتدا بیت مربوط به آخرین مقدار
        $1$
        مشاهده شده را انتخاب می‌کنیم. با توجه به اینکه ملزم به داشتن
        $r+1$
        بیت دارای مقدار
        $1$
        هستیم، جایگاه آخرین مقدار
        $1$
        می‌تواند
        $j+1$امین
        بیت باشد به نحوی که :
        $r+1 \leq j+1 \leq n+1$.
        حال بیت‌های اول تا
        $j$ام
        را درنظر بگیرید. در این رشته‌ی
        $j$ بیتی،
        ملزم به انتخاب
        $r$
        بیت برای
        $1$
        بودن هستیم که به
        $\binom{j}{r}$
        حالت صورت می‌گیرد. حال برای رسیدن به جواب مسئله، طبق اصل جمع، باید
        تعداد حالات ساختن رشته را به ازای تمام مقادیر ممکن
        $j$
        با هم جمع کنیم. بنابراین پاسح مسئله برابر است با :
        \begin{center}
            $\sum\limits_{j=r}^n \binom{j}{r}$
        \end{center}
        \end{enumerate}
        با توجه به یکتا بودن پاسخ مسئله‌ی طرح شده، مقدار بدست آمده از دو روش برابر است،
        پس داریم :
        \begin{center}
        $\binom{n+1}{r+1} = \sum\limits_{j=r}^n \binom{j}{r}$
        \end{center}
    }
\end{problem}