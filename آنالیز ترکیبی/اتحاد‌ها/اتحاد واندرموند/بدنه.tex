\SUBSECTION{اتحاد واندرموند}

\begin{fact}
    \focused{اتحاد واندرموند:}
    فرض کنیم
    $m,n,r$
    اعداد صحیح نامنفی باشند و
    $r \leq n$
    یا
    $r \leq m$
    باشد، داریم:
    \begin{center}
    $\binom{n+m}{r} = \sum\limits_{k=0}^r \binom{m}{r-k} \binom{n}{k}$
    \end{center}
\end{fact}

\begin{problem}[اثبات اتحاد واندرموند]
  اتحاد واندرموند را اثبات کنید.

  \problemsolution{
    برای اثبات این اتحاد از دوگانه شماری استفاده می‌کنیم. دو گروه از افراد داریم.
    گروه اول با اندازه‌ی
    $m$ نفر
    و گروه دوم با اندازه‌ی
    $n$
    نفر است.
    قصد داریم تعداد حالات انتخاب
    $r$
    نفر از این دو گروه را پیدا کنیم به تحوی که تعداد افراد انتخاب شده از هر گروه
    فاقد اهمیت باشد. برای این مقصود، دو شمارش زیر ممکن است :
    \begin{enumerate}
      \item 
      گروه‌بندی را فراموش کرده و از بین
      $m+n$
      نفر،
      $r$
      نفر را انتخاب می‌کنیم :
      \begin{center}
        $\binom{m+n}{r}$
      \end{center}

      \item 
      ابتدا تصمیم می‌گیریم چه تعداد از
      $r$
      نفر انتخابی از گروه اول باشند. این تعداد که می‌تواند مقدار صفر تا
      $r$
      بگیرد را 
      $k$
      می‌نامیم. بنابراین باید 
      $k$
      عضو از بین
      $n$
      عضو گروه اول و
      $r-k$
      عضو باقیمانده را از بین
      $m$
      نفر گروه دوم انتخاب کنیم. طبق اصل ضرب داریم :
      \begin{center}
        $\binom{n}{k} \binom{m}{r-k}$
      \end{center}
      با توجه به اصل ضرب، برای رسیدن به تعداد حالات کل، باید مقدار بالا را برای
      تمام مقادیر
      $k$
      با هم جمع کنیم :
      \begin{center}
        $\sum\limits_{k=0}^r \binom{n}{k} \binom{m}{r-k}$
      \end{center}
    \end{enumerate}
    با توجه به یکتا بودن پاسخ مسئله، مقدار به دست آمده از هر دو روش یکسان می‌باشند :
    \begin{center}
      $\binom{n+m}{r} = \sum\limits_{k=0}^r \binom{m}{r-k} \binom{n}{k}$
    \end{center}
  }
\end{problem}
