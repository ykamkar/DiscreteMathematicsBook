\begin{problem}[تعداد توابع پوشا]
  تعداد توابع پوشا از یک مجموعه
  $m$
  عنصری به یک مجموعه
  $n$
  عنصری
  را بیابید.
    
    \problemsolution{
      عملگر
      $P(i)$
      را تعداد توابع از مجموعه
      $m$
      عضوی به مجموعه
      $n$
      عضوی تعریف می‌کنیم به نحوی که حداقل
      $i$
      عضو از مجموعه مقصد پوشانده نشوند.
      اگر پاسخ مسئله (تعداد توابع پوشا) را
      $N$
      و تعداد کل توابع غیر پوشا از مجموعه
      $m$
      عضوی به مجموعه
      $n$
      عضوی را
      $A$
      بنامیم، طبق اصل متمم داریم :
      \begin{center}
        $N = P(0) - A$
      \end{center}
      همچنین طبق اصل شمول و عدم شمول داریم :
      \begin{center}
        $A = P(1) - P(2) + P(3) - ... + (-1)^{n-1} P(n) $
        
        $= \sum\limits_{i=1}^n (-1)^{i-1} P(i)$
      \end{center}
      
      برای بدست آوردن مقدار
      $P(i)$
      ابتدا 
      $i$
      عضو از مجموعه مقصد به عنوان اعضای پوشانده نشده
      انتخاب و آن‌ها را نادیده می‌گیریم.
      سپس تعداد کل توابع ممکن از مجموعه
      $m$
      عضوی مبدا به مجموعه
      $n - i$
      عضوی مقصد جدید را محاسبه می‌کنیم.
      طبق اصل ضرب، 
      $P(i)$
      برابر حاصل ضرب تعداد حالات انتخاب
      $i$
      عضو و تعداد توابع بر روی مجموعه مقصد جدید می‌باشد :
      \begin{center}
        $P(i) = \binom{n}{i} (n-i)^m$
      \end{center}
  
      بنابراین داریم :
      \begin{center}
        $N = P(0) - A = P(0) - \sum\limits_{i=1}^n (-1)^{i-1} P(i) = \sum\limits_{i=0}^n (-1)^i P(i)$
  
        $= \sum\limits_{i=0}^n (-1)^i \binom{n}{i} (n-i)^m$
      \end{center}

      \begin{theorem}
        تعداد توابع پوشا از یک مجموعه 
        $m$
        عضوی به یک مجموعه
        $n$
        عضوی برابر است با:
        $$\sum\limits_{i=0}^n (-1)^i \binom{n}{i} (n-i)^m$$
      \end{theorem}
    }
  
\end{problem}