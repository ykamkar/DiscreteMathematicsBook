\SUBSECTION{جایگشت خطی با اعضای تکراری}

\begin{problem}
    فرض کنید
    A
    یک جعبه حاوی
    $c_i$
    مهره از نوع
    $t_i$
    باشد و 
    $i \leq n , i \in \mathbb{N}$.
    تعداد جایگشت‌های خطی این اعضا را محاسبه کنید.

    \problemsolution{
        ابتدا تمام اعضا را متمایز متصور می‌شویم. تعداد کل مهره‌های درون
        A
        را $m$ می‌نامیم که:
        $$m = \sum\limits_{i=1}^n c_i \times t_i$$
        می‌دانیم تعداد جایگشت‌های خطی این اعضا (درصورتی که تمام مهره‌ها را از هم متمایز متصور شویم)
        برابر است با $m!$.
        حال می‌دانیم که هر
        $c_i!$
        جایگشت در این جایگشت‌ها، حاصل جایگشت‌های مختلف مهره‌های 
        $t_i$
        در مکان‌های یکسان است که اگر بخواهیم این مهره‌ها را نامتمایز بدانیم،
        یکسان خواهند بود. بنابراین اگر مهره‌های از نوع
        $t_i$
        را نامتمایز فرض کنیم، طبق اصل تقسیم، تعداد جایگشت‌ها برابر است با:
        $$\frac{m!}{c_i!}$$
        طبق استدلال مشابه، اگر تمام مهره‌های هم نوع را نامتمایز متصور شویم،
        طبق اصل تقسیم، تعداد جایگشت‌ها برابر است با:
        $$\frac{m!}{\prod\limits_{i=1}^n c_i!}$$
    }
\end{problem}

\begin{problem}
    با حروف
    t,t,a,c,a,t,s
    چند رشته حرفی متمایز به طول ۷ می‌توان ساخت؟

    \problemsolution{
        $$\frac{7!}{3!2!1!1!} = 420$$
    }
\end{problem}