    بهزاد که یک دانشجوی نمونه است؛ این ترم برای خود برنامه‌ریزی کرده است که از شروع ترم، هر هفته ۲۱ ساعت درس بخواند و ۱۵ ساعت بازی کند. بهزاد می‌تواند با تبلت، گوشی و یا لپ‌تاپ خود درس بخواند؛ در حالیکه برای بازی کردن، به لپ‌تاپ و یا کنسولش نیاز دارد. برای صرفه‌جویی در مصرف برق، بهزاد تصمیم گرفته است که این چهار وسیله‌ی الکترونیکی را فقط یک بار در اول هفته شارژ کند.
    
    اگر بدانیم که تبلت و گوشی او ۱۰ ساعت، لپ‌تاپ او فقط ۸ ساعت و کنسولش ۱۲ ساعت شارژ نگه می‌دارد؛ بهزاد به چند طریق می‌تواند وقت خود را بین این چهار دستگاه تقسیم کند؟(از آنجایی که بهزاد می‌خواهد برنامه‌ریزی‌اش تا جای
ممکن ساده باشد، در هر ساعت از روز فقط از یکی از وسایل استفاده می‌کند و نمی‌خواهد بازه‌های استفاده‌ی خود را به دقیقه تقسیم کند.)
