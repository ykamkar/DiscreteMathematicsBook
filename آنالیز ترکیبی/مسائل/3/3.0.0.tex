  \begin{enumerate}
    \item 
    تعداد جایگشت‌های حروف غیر از m و n طبق جایگشت حروف با تکرار، برابر $\frac{6!}{2!2!}$ است.
    (هر یک از حروف a و e دوبار تکرار شده‌اند)
    حال جایگاه حروف m و n را تعیین می‌کنیم.
    برای این کار، چهار حالت داریم:
    \begin{enumerate}
      \item 
      حالتی که در آن هر کدام از عبارات ma و ne
      دوبار تکرار شوند.بدین منظور، دو جایگاه برای قرار دادن دو حرف m داریم:
      قبل از حرف  a که دو بار در عبارت تکرار شده است.(1)
      
      هم‌چنین دو جایگاه برای قرار دادن دو حرف n داریم:
      قبل از حرف e که دو بار در عبارت آمده است.
      (2)
      
      بدین ترتیب تعداد حالات ممکن موجود در این حالت به صورت زیر خواهد بود:
      $$\underbrace{{2 \choose 2}}_{1}\underbrace{{2 \choose 2}}_{2}$$
      \item
      حالتی که در آن عبارت ma دو بار و عبارت ne یک بار آمده باشد. در این صورت مانند قسمت آ دو جایگاه برای 
      برای دو حرف m داریم:
      قبلا از حرف a که دو بار در عبارت تکرار شده است.
      (1)
      
      طبق حالت بیان شده، یکی از حروف n باید قبل از یکی از دو حرف e
      بیاید. در این حالت دو انتخاب برای قرار دادن یک حرف n داریم.(2)
      
      و در نهایت شش حالت برای قرار دادن حرف دوم n در عبارت خواهیم داشت.
      (دقت شود که دو عبارت ma و عبارت ne موجود در حروف
      در واقع یک کلمه محسوب می‌شود که همراه حروف  g و t باقی‌مانده، ۶ انتخاب در اختیار ما خواهند گذاشت.
      لازم به ذکر است که ما مجاز نیستیم که این حرف n را قبل از حرف e باقی‌مانده در جایگشت حروف قرار دهیم چرا که به حالت آ می‌رسیم که یک بار آن را شمرده‌ایم.)
      (3)
      
      پس تعداد جایگشت‌های ممکن در این حالت به این صورت خواهد بود:
      $$\underbrace{{2\choose 2}}_{1}\underbrace{{2 \choose 1}}_{2}\underbrace{{6\choose 1}}_{3}$$
      \item
      حالت سوم مشابه حالت دوم می‌باشد با این تفاوت که این بار دو بار 
      عبارت ne و یک بار عبارت ma آمده باشد.
      نحوه‌ی شمارش تعداد حالات ممکن برای قرار دادن حرف n مشابه حالات قرار دادن حرف m در حالت ب است.(1)
      
      هم‌چنین حالات ممکن برای قرار دادن حرف m مشابه قرار دادن حرف n در حالت ب می‌باشد با این تفاوت که این بار مجاز به قرار دادن هر دو حرف n قبل از حرف e نمی‌باشیم.(2) (3)
      
      تعداد نهایی مطابق زیر خواهد بود:
      $$\underbrace{{2\choose 2}}_{1}\underbrace{{2 \choose 1}}_{2}\underbrace{{6\choose 1}}_{3}$$
      \item
      حالت نهایی مختص به زمانی است که هریک از عبارات ma و ne تنها یک بار آمده باشد.
      بدین منظور ابتدا یکی از دو حرف e موجود را برای قرار دادن حرف n قبل از آن انتخاب می‌کنیم.(1)
      
      سپس یکی از دو حرف a موجود را برای قرار دادن حرف m قبل از آن انتخاب می‌کنیم.(2)
      
      برای قرار دادن حرف m باقی‌مانده، ۶ حالت داریم.
      قبل و بعد همه‌ی عبارات و حروف 
      ma 
      ne، t، g، e،
      می‌توانیم حرف m راقرار دهیم.
      دقت شود که قبل از حرف a مجاز به قرار دادن حرف m نیستیم چرا که این حالت قبلا شمرده شده است.(3)
      
      در نهایت تعداد حالات ممکن برای قرار دادن حرف n را
      می‌شماریم که ۷ تاست. می‌توانیم قبل و بعد عبارات
      ma
      ne،
      m،
      t، g، a،
      حرف n را قرار دهیم.
      قابل ذکر است که در این حالت نمی‌توان قبل حرف e
      حرف n را قرار داد چون این حالت قبلا شمرده شده است.
      (4)
      
      بدین ترتیب حالت نهایی به صورت زیر خواهد بود:
      $$\underbrace{{2\choose 1}}_{1}\underbrace{{2\choose 1}}_{2}\underbrace{{6\choose 1}}_{3}\underbrace{{7\choose 1}}_{4}$$
    \end{enumerate}
    پس تعداد نهایی برای قرار دادن حروف m و n به طریق زیر می‌باشد:
    $$\underbrace{{2 \choose 2}{2 \choose 2}}_{\text{آ}} + \underbrace{{2\choose 2}{2 \choose 1}{6\choose 1}}_{\text{ب}} + \underbrace{{2\choose 1}{2\choose 2}{6\choose 1}}_{\text{ج}}$$
    $$+ \underbrace{{2\choose 1}{2\choose 1}{6\choose 1}{7\choose 1}}_{\text{د}} = 193$$

    در نهایت پاسخ نهایی برابر $\frac{6!}{2!2!} \times 193$ است.
    \item
    ابتدا همه‌ی حروف غیر از دو حرف m را مطابق شرایط سوال می‌چینیم.
    از میان ۸ جایگاه موجود برای ۸ حرف، ۴ جایگاه را برای چیدن حروف صدادار به ترتیب الفبایی می‌چینیم که فقط به یک صورت انجام می‌پذیرد.(1)
    
    سپس برای چهار حرف صامت باقی‌مانده، یعنی دو حرف n،
    t و g
    طبق جایگشت حروف باتکرار در ۴ جایگاه باقی‌مانده قرار می‌دهیم.(2)
    
    $$\underbrace{{8\choose 4}}_{1} \times \underbrace{\frac{4!}{2!}}_{2}$$
    حال به دو صورت می‌توانیم دو حرف m را قرار دهیم به نحوی که حداقل یک عبارت 
    ma تولید شود:
    \begin{enumerate}
      \item 
      دو عبارت ma در جایگشت داشته باشیم. بدین ترتیب تنها کافیست در دو جایگاه قبل از حرف a دو حرف m را قرار دهیم.(1)
      \item
      فقط یک عبارت ma داشته‌باشیم. در این صورت ابتدا نیاز است از بین دو جایگاه موجود قبل دو حرف a یکی را برای قرار دادن حرف m انتخاب کنیم.(2)
      
      سپس از بین جایگاه‌های باقیمانده یک جایگاه را برای قرار دادن دومین m انتخاب کنیم. دقت شود که مجاز به قرار دادن m قبل از حرف a باقی‌مانده نیستیم چرا که این حالت در قسمت آ شمرده شده است. هم‌چنین عبارت ma یک حرف در نظر گرفته شده و قبل و بعد آن هر کدام یک جایگاه به حساب می‌آید.(3)
    \end{enumerate}
    در نهایت تعداد حالات موجود برای قرار دادن حرف m به صورت زیر می‌باشد:
    $$\underbrace{2\choose 2}_{1} + \underbrace{2\choose 1}_{2}\underbrace{8\choose 1}_{3} = 17$$
    
    در نهایت پاسخ نهایی به صورت زیر خواهد بود:
    $${8\choose 4} \times \frac{4!}{2!} \times 17$$
    \item
    ابتدا تعداد جایگشت‌های  حروف غیر از a که اولین m قبل از اولین n و چسبیده به آن آمده‌باشد را می‌شماریم.
    به این منظور عبارت mn را یک عبارت مستقل در نظر می‌گیریم.
    حال به غیر از دو حرف m و n و عبارت mn، 
    چهار حرف داریم که در آن e دوبار تکرار شده است.
    از هفت جایگاه موجود برای کلمات، چهار جایگاه آن را برای قرار دادن این حروف انتخاب می‌کنیم(1)
    
    سپس طبق جایگشت باتکرار، آن‌ها را در جایگاه‌های انتخاب شده می‌چینیم.(2)
    
    حال در میان سه جایگاه باقیمانده، طبق خواسته‌ی سوال، جایگاه اول متعلق به mn می‌باشد
    (اولین حرف m قبل از اولین حرف n وچسبیده به آن قرار داشته باشد)
    و در دو جایگاه باقی‌مانده دو حرف m و n را به دو صورت می‌توانیم قرار دهیم.(3)\\
    بدین ترتیب حالات قرار دادن حروف به غیر از a به صورت زیر خواهد بود:
    $$\underbrace{{7\choose 4}}_{1} \times \underbrace{\frac{4!}{2!}}_{2} \times \underbrace{2}_{3}$$
    
    حال باید حرف a را در عبارت قرار دهیم.
    به این منظور مشابه قسمت قبل، دو حالت داریم:
    \begin{enumerate}
      \item 
      هر دو حرف a بعد از دو حرف n قرار داشته باشد. در این صورت کافیست دو حرف a را بعد از دو حرف n موجود در جایگشت قرار دهیم.(1)
      \item
      فقط یک عبارت na در کلمه داشته‌باشیم. در این صورت ابتدا از میان دو جایگاه موجود پس از دو حرف n در کلمه، یکی را انتخاب می‌کنیم و یک حرف a را قرار می‌دهیم. سپس از میان ۷ جایگاه ممکن، یکی را برای حرف n باقی‌مانده انتخاب می‌کنیم.
      توجه شود که عبارت mn و na ایجاد شده را یک کلمه‌ در نظر می‌گیریم. هم‌چنین مجاز به قرار دادن حرف a پس از حرف n باقی‌مانده نیستیم چرا که این حالت در قسمت آ شمرده شده است.(2)
    \end{enumerate}
    بنابراین تعداد حالات قرار دادن حرف a به صورت زیر می‌باشد:
    $$\underbrace{2\choose 2}_{1} + \underbrace{{2\choose 1}{7\choose 1}}_{2} = 15$$ 
    پس پاسخ نهایی برابر است با:
    
    $${7\choose 4} \times \frac{4!}{2!} \times 2 \times 15$$
    \item
    تعداد جایگشت حروف غیر از m
    طبق حروف باتکرار به صورت زیر خواهد بود:
    (حروف n و e وa هر کدام دوبار تکرار شده‌اند.)
    $$\frac{8!}{2!2!2!}$$
    
    از طرفی تعداد جایگشت‌هایی از میان جایگشت‌های فوق که عبارت gn داشته باشد را محاسبه می‌کنیم. برای این کار کافیست عبارت gn را یک حرف در نظر بگیریم. حال طبق جایگشت حروف باتکرار، دو حرف a و e هرکدام دوبار تکرار شده‌اند. پس تعداد جایگشت‌هایی که این شرایط را داشته‌باشند به صورت زیر است:
    $$\frac{7!}{2!2!}$$
    
    حال به جایگذاری دو حرف m باقی‌مانده می‌پردازیم. برای این کار، دو حالت داریم:
    \begin{enumerate}
      \item 
      عبارت gn در کلمه‌ی فعلی نباشد:
      
      در این صورت، یا یکی از ۸ جایگاه ممکن را انتخاب می‌کنیم و هر دو حرف m را دقیقا کنار هم می‌گذاریم.
      (1)
      
      و یا اینکه از ۸ جایگاه موجود، دو جایگاه را انتخاب کرده و دو حرف m را قرار می‌دهیم. دقت شود که دو حرف یکسان می‌باشد و ترتیب جایگذاری آن اهمیتی ندارد.(2)
      قابل ذکر است که طبق خواسته‌ی سوال نباید عبارت  mg در جایگشت موجود داشته‌باشیم، به همین دلیل، ۸ انتخاب داریم.(قبل و بعد تمام حروف موجود به جز قبل از حرف g در جایگشت )
      $$\underbrace{8\choose 1}_{1} + \underbrace{8\choose 2}_{2}$$
      \item
      عبارت gn در کلمه‌ی فعلی وجود داشته‌باشد:
      
      در این صورت حتما باید یک حرف m میان دو حرف 
      g و n قرار دهیم تا عبارت gn دیگر وجود نداشته باشد.
      سپس از ۸ جایگاه باقی‌مانده یک جایگاه را برای قرار دادن حرف دوم m انتخاب می‌کنیم. دقت شود تعداد انتخاب‌های موجود برای قرار دادن حرف m هشت‌تاست چرا که علاوه بر آنکه قبل حرف g نمی‌توانیم حرف m را قرار دهیم، قرار دادن آن قبل و بعد حرف m موجود در کلمه، یک حالت به شمار می‌آید. پس در نهایت ۸ انتخاب خواهیم داشت.
      $${8\choose 1}$$
    \end{enumerate}
    بدین ترتیب پاسخ نهایی به صورت زیر خواهد بود:
    $$(\frac{8!}{2!2!2!} - \frac{7!}{2!2!})({8\choose 1} + {8\choose 2}) + \frac{7!}{2!2!} \times {8\choose 1}$$
  \end{enumerate}
