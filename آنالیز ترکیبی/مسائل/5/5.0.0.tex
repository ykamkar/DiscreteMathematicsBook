    تعداد راه‌های قرار دادن
    $n+1$
    شی متمایز در
    $k+1$
    دسته یکسان به صورتی که هیچ دسته‌ای خالی نماند($\genfrac{\{}{\}}{0pt}{}{n + 1}{k + 1}$) را به صورت زیر می‌شماریم:
    
    در ابتدا انتخاب می کنیم که کدام اشیا با شی اول در یک دسته قرار دارند، اگر تعداد این اشیا $j, (0 \leq j \leq n - k)$ باشد،
    تعداد اشیای باقی‌مانده جهت قرار دادن در سایر دسته‌ها برابر
    $i$
    خواهد بود که به صورت
    $i= n-j ,(k \leq i \leq n)$
    تعریف می‌شود و این انتخاب
    به ازای هر $j$، ${n\choose j}={n \choose n-j}={n\choose i}$ حالت دارد.
    
	قرار دادن $n-j$ شی باقی مانده در دسته ها هم $\genfrac{\{}{\}}{0pt}{}{n-j}{k}=\genfrac{\{}{\}}{0pt}{}{i}{k}$ حالت خواهد داشت.

    
	در نهایت با جمع زدن همه حالات ممکن برای $i$ به جواب مسئله می‌رسیم:
    
    $$\genfrac{\{}{\}}{0pt}{}{n + 1}{k + 1} = \sum\limits_{i=k}^{n} {n\choose i} \genfrac{\{}{\}}{0pt}{}{i}{k}$$
