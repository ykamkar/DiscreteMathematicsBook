\SUBSECTION{سری}
\begin{definition}
    \p
    شرکت‌پذیری عمل جمع روی مجموعه اعداد حقیقی (مختلط) موجب می‌شود که مجموعه‌ای متناهی به صورت
    $a_1 + a_2 + \ldots + a_n$
    دارای معنی بوده و بدون ابهام باشند.
    \p
   دنباله
$a_n$
   را درنظر بگیرید دنباله جدید
$S_n$
   را به صورت زیر تعریف می‌کنیم:
   $$S_1 = a_1$$
   $$ S_2 = a_1 + a_2$$
   $$\vdots$$ 
   $$S_n = a_1 + a_2 + \ldots $$
   \p
   $S_n$
   را یک سری می‌نامیم و آن را به صورت
   $\sum_{n=1}^{\infty}$
   نشان می‌دهیم.
   $a_n$
   را جمله عمومی سری و
   $S_n$
   را مجموع جزئی
   $n$
   ام آن می‌نامیم.
   \p
   باید توجه داشت که
   $S_n$
   مجموع
   $n$
   جمله اول سری است و به اینکه 
   $n$
   از 
   $0$
   یا
   $1$
و یا هر عدد دیگری شروع شده باشد، بستگی ندارد.
\end{definition}
	
	
\begin{definition}
    \p
    \focused{همگرایی و عدم‌همگرایی سری‌ها}
    سری
   $\sum_{n=1}^{\infty}$
را در صورتی که دنباله مجموع‌های جزئی آن همگرا باشد، 
	\focused{همگرا}
، در غیر اینصورت 
	\focused{واگرا} 
 نامیده می‌شود.
\end{definition}
	