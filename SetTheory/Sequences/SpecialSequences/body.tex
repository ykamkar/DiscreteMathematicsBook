\SUBSECTION{دنباله‌های خاص}
\begin{definition}
    \p
    \focused{دنباله حسابی}
    یا تصاعد حسابی به دنباله‌ای از اعداد گفته می‌شود که اختلاف هر دو جمله‌ی متوالی آن مقداری ثابت باشد. به این مقدار ثابت 
   \focused{قدر نسبت دنباله حسابی}
      گفته می‌شود.
\end{definition}
	\p
به طور مثال دنباله‌ی زیر، یک دنباله‌ حسابی می‌باشد:
$$2, 5, 8, 11, 14, ...$$
	\p
اگر جمله‌ی نخست دنباله‌ای حسابی برابر با
$a_1$
 و قدرنسبت آن برابر با
$d$
باشد، جمله‌ی عمومی این دنباله به‌صورت 
$a_n = a_1 + (n - 1)d$
است. مجموع
$n$
جمله‌ی نخست این دنباله‌ی حسابی برابر است با:
$$S_n = \frac{n}{2}(a_1 + a_n)$$
با توجه به:
$$a_n = a_1 + (n - 1)d$$
داریم:
$$S_n = \frac{n}{2}(2a_1 + (n - 1)d)$$
	\p
اگر
$S_n$
مجموع
$n$
جمله‌ی نخست دنباله‌ای حسابی باشد، آن‌گاه
$S_1 = a_1$
و
$S_{n+1} - S_n = a_{n+1}$
می‌باشد.
	\p
اگر
$S_n$
مجموع
$n$
جمله‌ی نخست دنباله‌ای حسابی باشد، عددهایی حقیقی و یکتا مانند
$A$
و
$B$
وجود دارند که:
$$S_n = An^2 + Bn$$
	\p

\subfile{./example1.tex}
\subfile{./example2.tex}

\begin{definition}
    \p
    \focused{دنباله هندسی}
   دنباله‌ای از اعداد است که نسبت هر دو جمله متوالی آن مقداری ثابت باشد.به این مقدار ثابت 
   \focused{قدر نسبت دنباله هندسی}
  گفته می‌شود.    
\end{definition}
	\p
به طور مثال دنباله زیر، یک دنباله هندسی می‌باشد.
	\p
$$2, 6, 18, 54, ...$$
	\p
اگر جمله‌ی نخست دنباله‌ای هندسی برابر
$a_1$
و قدر‌نسبت آن برابر با
$q$
باشد، جمله‌ی عمومی این دنباله به‌صورت 
$a_n = a_1q^{n-1}$
 است. مجموع 
$n$
جمله‌ی نخست این دنباله‌ی هندسی برابر است با:
$$S_n = a_1\frac{q^n - 1}{q - 1}$$

	\p
اگر
$S_n$
مجموع
$n$
جمله‌ی نخست دنباله‌ای هندسی باشد، آن‌گاه
$S_1 = a_1$
و
$S_{n+1} - S_n = a_{n+1}$
می‌باشد.
	\p
اگر
$a$
عددی حقیقی و 
$n$
 عددی طبیعی باشد، آن‌گاه:
$$a^n - 1 = (a - 1)(a^{n-1} + a^{n-2} + \cdots + a + 1)$$

\subfile{./example3.tex}


\begin{definition}
    \p
    \focused{دنباله فیبوناچی}
    دنباله‌ای از اعداد است که در آن به جز دو جمله اول، هر جمله از مجموع دو جمله‌ی قبلی به دست می‌آید.
    \p
  $$1, 1, 2, 3, 5, 8, 13, 21, 34, 55, 89, ...$$
	\p 
\end{definition}
\begin{extra}{دنباله فیبوناچی}
دنباله فیبوناچی خواص شگفت‌انگیزی دارد که باعث می شود در آثار برجسته هنر و معماری، دانه‌های گل آفتاب گردان، صدف‌ها و تولید مثل خرگوش‌ها، آثار آن دیده شود. به زودی مطالب بیش‌تری در این قسمت قرار خواهد گرفت.
\end{extra}


