\SUBSECTION{دنباله‌های خاص}
\begin{definition}
    \p
    \focused{دنباله حسابی}
    یاتصاعد حسابی به دنباله‌ای از اعداد گفته می‌شود که اختلاف هر دو جمله متوالی آن مقداری ثابت باشد. به این مقدار ثابت 
   \focused{قدر نسبت}
     دنباله حسابی گفته می‌شود.
\end{definition}
	\p
به طور مثال دنباله زیر، یک دنباله‌ حسابی می‌باشد.
	\p
$$2, 5, 8, 11, 14, ...$$
	\p
سری حسابی که دنباله آن از
$a$
شروع شود و قدر نسبت آن
$d$
 و
$n$
عضو داشته باشد، برابر است با:
	\p
$$\frac{n(n-1)}{2}d + a_n$$
	\p


\subfile{./example1.tex}
\subfile{./example2.tex}
\subfile{./example3.tex}


\begin{definition}
    \p
    \focused{دنباله هندسی}
   دنباله‌ای از اعداد است که نسبت هر دو جمله متوالی آن مقداری ثابت باشد.به این مقدار ثابت 
   \focused{قدر نسبت}
    دنباله هندسی گفته می‌شود.    
\end{definition}
	\p
به طور مثال دنباله زیر، یک دنباله هندسی می‌باشد.
	\p
$$2, 6, 18, 54, ...$$
	\p
	سری هندسی که دنباله آن از
$a$
شروع شود و قدر نسبت آن
$d$
 و
$n$
عضو داشته باشد، برابر است با:
	\p
$$\frac{d^n - 1}{d - 1}a$$
	\p


\subfile{./example4.tex}


\begin{definition}
    \p
    \focused{دنباله فیبوناچی}
    دنباله‌ای از اعداد است که در آن به جز دو جمله اول، هر جمله از مجموع دو جمله قبلی به دست می‌آید.
    \p
  $$1, 1, 2, 3, 5, 8, 13, 21, 34, 55, 89, ...$$
	\p 
\end{definition}
\begin{extra}{دنباله فیبوناچی}
دنباله فیبوناچی خواص شگفت‌انگیزی دارد که باعث می شود در آثار برجسته هنر و معماری، دانه‌های گل آفتاب گردان، صدف‌ها و تولید مثل خرگوش‌ها، آثار آن دیده شود.
\end{extra}
