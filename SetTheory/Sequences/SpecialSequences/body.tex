\SUBSECTION{دنباله‌های خاص}
\p
\begin{definition}
    \p
    \focused{دنباله حسابی}
    یاتصاعد حسابی دنباله ای از اعداد است که اختلاف هر دو جمله متوالی آن مقداری ثابت است. به این مقدار ثابت قدر نسبت تصاعد حسابی گفته می شود.
    مثل:
    $2, 5, 8, 11, 14, ...$
\end{definition}
\p

\subfile{./example1.tex}
\subfile{./example2.tex}
\subfile{./example3.tex}
\p
\begin{definition}
    \p
    \focused{دنباله هندسی}
   دنباله ای از اعداد است که نسبت هر دو جمله متوالی آن مقداری ثابت باشد.به این مقدار ثابت قدر نسبت تصاعد هندسی گفته می شود.
    مثل:
  $2, 6, 18, 54, ...$  
\end{definition}
\p

\subfile{./example4.tex}
\subfile{./example5.tex}
\subfile{./example6.tex}
\p
\begin{definition}
    \p
    \focused{دنباله فیبوناچی}
    دنباله ای از اعداد است که در آن به جز دو جمله اول هر جمله از مجموع دو جمله قبلی به دست می آید.
    \p
  $$1, 1, 2, 3, 5, 8, 13, 21, 34, 55, 89, ...$$
	\p
  دنباله فیبوناچی خواص شگفت انگیزی دارد که باعث می شود در آثار برجسته هنر و معماری تا دانه های گل آفتاب گردان و صدف ها و تولید مثل خرگوش ها آثار آن دیده شود. 
\end{definition}
