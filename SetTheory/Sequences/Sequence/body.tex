\SUBSECTION{دنباله}
\begin{definition}
	\p
\focused{دنباله}
تابعی است که دامنه‌ی آن مجموعه‌ی اعداد حسابی یا طبیعی و بردش مجموعه‌ای غیرتهی مانند
 $A$
 باشد. اعداد واقع در برد یک دنباله را جملات دنباله و جمله 
 $n$ 
 ام آن را با
 $a_n$
 نمایش داده و 
\focused{جمله عمومی}
  دنباله می‌نامیم. بنابراین اگر تابع
$f$
  از
$N$
  به
$A$ 
  یک دنباله و مقدار 
$f$
  بر حسب
$n$
    باشد، آن را به‌صورت
$f_(n) = a_n$
  نمایش می‌دهیم.
\end{definition}

	\p
به زبان ساده‌تر به اعدادی که به تعداد متناهی یا نامتناهی دارای ترتیب باشند، دنباله گوییم. در دنباله‌ها تکرار مجاز است و ترتیب اهمیت دارد.
باید توجه داشت که در برخی منابع دامنه‌ی دنباله‌ها تنها اعداد طبیعی در نظر گرفته شده است.

\begin{definition}
\begin{enumerate}

\item
 دنباله
$a_n$
\focused{صعودی}
  نامیده می‌شود اگر به ازای
$n \in \mathbb{N}$
 داشته باشیم:
$$a_{n+1} > a_n$$

\item
 دنباله
$a_n$
\focused{نزولی}
  نامیده می‌شود اگر به ازای
$n \in \mathbb{N}$
 داشته باشیم:
$$a_{n+1} < a_n$$

\item
 دنباله
$a_n$
\focused{ناصعودی}
  نامیده می‌شود اگر به ازای
$n \in \mathbb{N}$
 داشته باشیم:
$$a_{n+1} \leq a_n$$

\item
 دنباله
$a_n$
\focused{نانزولی}
  نامیده می‌شود اگر به ازای
$n \in \mathbb{N}$
 داشته باشیم:
$$a_{n+1} \geq a_n$$

\item
 دنباله‌ای حقیقی  که دارای یکی از ویژگی‌های بالا است، دنباله 
 \focused{یکنوا}
 نامیده می‌شود.

\item
 دنباله
$a_n$
 را  
 \focused{از بالا کراندار}
  می‌نامند اگر عدد مثبت 
 $M$
 وجود داشته باشد که به ازای هر
$n \in \mathbb{N}$
 داشته باشیم:
$$a_{n} \leq M$$

\item
 دنباله
$a_n$
 را 
 \focused{از پایین کراندار}
  می‌نامند اگر عدد مثبت 
 $M$
 وجود داشته باشد که به ازای هر
$n \in \mathbb{N}$
 داشته باشیم:
$$a_{n} \geq M$$

\item
 دنباله
$a_n$
\focused{کراندار}
 نامیده می‌شود اگر هم از بالا و هم از پایین کراندار باشد.

\item
دنباله‌ای که کراندار نباشد 
\focused{بی‌کران}
 است.
\end{enumerate}
\end{definition}


\SUBSECTION{همگرایی یا عدم‌همگرایی دنباله‌ها}
\begin{definition}
	\p
دنباله عددی
$a_n$
 به عدد 
$L$
\focused{همگرا}
است اگر به ازای هر
$\epsilon > 0$
عدد طبیعی
$N$
وجود داشته باشد به طوری که:
$$n > N \Rightarrow |a_n - L| < \epsilon$$
به عبارت دیگر دنباله فوق به عدد
$L$
همگرا است اگر به ازای هر
$\epsilon > 0$
از مرحله‌ای به بعد تمام جمله‌های آن در
$\epsilon$
 همسایگی
$L$
 قرار گیرند. دنباله‌ای که به عددی همگرا نباشد، 
\focused{واگرا}
  نامیده می‌شود. در حقیقت همگرایی دنباله 
$a_n$
به عدم 
$L$
هم‌ارز تعریف عدد
$L$
به عنوان حد در بی‌نهایت تابعی است که دنباله را تعریف می‌کند و چون حد تابع در هر نقطه منحصر به فرد است. پس
$L$
یکتاست.
\end{definition}
	
\SUBSECTION{انواع دنباله‌‌های همگرا}
\begin{theorem}
	\p
هر دنباله‌ی یکنوا و کراندار، همگرا است. از مهم‌ترین ویژگی‌های دنباله‌های همگرا کرانداربودن آن‌هاست. بنابراین دنباله‌های همگرا زیردسته‌ای از دسته دنباله‌های کراندار هستند. عکس این مطلب صحیح نیست یعنی دسته دنباله‌های کراندار زیردسته دنباله‌های همگرا نیست. با توجه به مطالب ذکر شده نتیجه مهم دیگری که می‌گیریم این است که: هر دنباله همگرا کراندار است. اما ممکن است دنباله‌ای کراندار باشد ولی همگرا نباشد مثل دنباله
$a_n = (-1)^n$
با اینکه کراندار است ولی واگراست. توجه می‌کنیم که در کاربرد قضیه ذکر شده در بالا باید هر دو شرط یکنوایی و کرانداری هم‌زمان برقرار باشد تا نتیجه بگیریم دنباله همگراست. در مثال ذکر شده دنباله یکنوا نیست زیرا به ازای
$n$
های مثبت پاسخ مثبت
$1$
می‌شود و به ازای 
$n$
های فرد پاسخ 
$-1$
خواهد بود پس یکنوا نیست بلکه نوسانی است بنابراین حد ندارد در نتیجه واگراست.
\end{theorem}
	\p
	
\begin{NOTE}
دنباله‌های ثابت همگرا هستند یعنی اگر
$k$
عدد ثابت دلخواهی باشد آنگاه دنباله ثابت 
$k$
 که به ازای هر 
$n$
 با 
$a_n = k$
  تعریف شده است همگرا به 
$k$
می‌باشد.
\end{NOTE}

\begin{definition}
	\p
	\focused{دنباله‌های کشی}
دنباله 
$a_n$
را کشی گویند اگر به ازای هر
$\epsilon > 0$
عدد طبیعی
$N$
وجود داشته باشد که:
$$m > n, n > N \Rightarrow |a_n - a_m| < \epsilon$$
	\p
نکته بسیار مهم درباره دنباله‌های کشی این است که هر دنباله کشی همگراست. عکس این مطلب نیز صحیح است یعنی هر دنباله کشی همگراست. این مطلب را بدون اثبات می‌پذیریم.
\end{definition}

\begin{NOTE}
	\p
هرگاه دنباله‌های
$a_n$
و
$b_n$
 به ترتیب به
$A, B$
همگرا باشند آنگاه مجموع دو دنباله به
$(A + B)$
 همگرا است. ضرب دو دنباله فوق در یکدیگر به 
$(A.B)$
همگراست. حاصل تقسیم دو دنباله ذکر شده به 
$(\frac{A}{B})$
 همگراست مشروط بر اینکه 
$B \neq 0$
 و
$b_n$
هرگز صفر نباشد. هرگاه
$k$
 یک عدد ثابت و دلخواه باشد در این صورت
$\lim ka_n = kA$
فرض است که جمیع حدود به ازای
$n$
به سمت بی‌نهایت گرفته می‌شوند.
\end{NOTE}

\begin{NOTE}
	\p
هرگاه دنباله
$a_n$
واگرا بوده و
$C$
 عددی مخالف صفر باشد آنگاه دنباله 
$Ca_n$
 واگرا می‌باشد.
\end{NOTE}

\begin{definition}
	\p
\focused{قضیه ساندویچ}
هرگاه به ازای هر 
$n$
بزرگتر از اندیسی چون 
$N$
،
$a_n \leq b_n \leq c_n$
 و
$\lim a_n = \lim c_n = L$
آنگاه نیز
$\lim b_n = L$
خواهد بود. کاربرد مطالب فوق توسط قضیه‌ای وسیع می‌شود که می‌گوید حاصل اعمال یک تابع پیوسته بر یک دنباله واگرا ، دنباله‌های همگراست.
\end{definition}

\begin{theorem}
	\p
هرگاه
$a_n$
به
$L$
میل کند و تابع
$f$
در
$L$
پیوسته باشد و در جمیع 
$a_n$
ها تعریف شده باشد آنگاه:
$$f(a_n) \rightarrow f(L)$$
\end{theorem}

%	\englishblock{
%    \href{http://daneshnameh.roshd.ir/mavara/mavara-index.php?page=%d8%af%d9%86%d8%a8%d8%a7%d9%84%d9%87+	%d9%88+%d8%b3%d8%b1%db%8c&SSOReturnPage=Check&Rand=0}
%    {دنباله و سری}
%    \footnote{http://daneshnameh.roshd.ir/mavara/mavara-index.php?page=%d8%af%d9%86%d8%a8%d8%a7%d9%84%d9%87+%d9%88+%d8%b3%d8%b1%db%8c&SSOReturnPage=Check&Rand=0}
%	}

\subfile{./example1.tex}
\subfile{./example2.tex}
\subfile{./example3.tex}
