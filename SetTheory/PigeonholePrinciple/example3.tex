\begin{problem}[قضیه‌ Erdos-Zekeres]
  \p
  اگر دنباله‌ای به طول 
  $pq + 1$
  از اعداد حقیقی متمایز داشته باشیم، ثابت کنید در این دنباله، زیر‌دنباله‌ای اکیدا صعودی به طول 
  $p+1$
  یا اکیدا نزولی به طول 
  $q+1$
  وجود دارد. 

  \problemsolution{
    \p
    اگر 
    $1 \leq m \leq pq+1$
    باشد، می‌توانیم 
    $L_{m}$
          و 
    $R_{m}$
          را تعریف کنیم به طوری که 
    $L_{m}$      
            طول بزرگ‌ترین زیر‌دنباله‌ی اکیدا صعودی است که به $m$امین عضو دنباله ختم می‌شود و 
    $R_{m}$
            طول بزرگ‌ترین زیر‌دنباله‌ی اکیدا نزولی است که از عضو $m$ام دنباله شروع می‌شود. حال اگر $k$ را در نظر بگیریم به طوری که با $m$ مساوی نباشد، حتما یکی از دو رابطه‌ی
    $R_{m} \neq R_{k}$        
              یا 
    $L_{m} \neq L_{k}$
              برقرار است. (زیرا اگر 
    $m > k$          
                ، برحسب اینکه عضو $m$ام دنباله از عضو $k$ام دنباله بزرگ‌تر باشد یا کوچک‌تر، داریم 
    $L_{k} < L_{m}$
                یا 
    $R_{k} > R_{m}$   
                  . در حالت 
    $m < k$
                نیز  اتفاقات مشابهی رخ می‌دهد.) به این ترتیب تمام زوجهای
    $(L_{m}, R_{m})$
    که متفاوت هستند. 
    \p
    برهان خلف: زیر دنباله‌ای اکیدا صعودی به طول 
    $p+1$
      یا زیردنباله‌ای اکیدا نزولی به طول 
    $q+1$
      موجود نباشد.
      \p
        بنابراین 
    $1\leq L_{m}\leq p$
      و 
    $1\leq R_{m}\leq q$
    است. پس برای هر $m$، زوج 
    $(L_{m}, R_{m})$
      می‌تواند 
    $p \times q$
    حالت داشته باشد و از آن‌جایی که $m$ می‌تواند 
    $pq + 1$
      مقدار مختلف داشته باشد، طبق اصل لانه‌ی کبوتری، حداقل دو تا از زوج‌های 
    $(L_{m}, R_{m})$
      با هم برابرند که با متمایز بودن همه‌ی زوج‌های 
    $(L_{m}, R_{m})$
      در تناقض است. بنابراین برهان خلف نادرست است و درستی حکم ثابت می‌شود. 
  }
\end{problem}