\begin{problem}
  \p
    یک دسته‌ی ۵۲‌تایی از کارت‌ها داریم.
    \begin{enumerate}
      \item 
      چه تعداد از کارت‌ها را باید انتخاب کنیم تا دست کم ۳ تا از کارت‌ها هم‌خال باشند؟
  
      \problemsolution{
        \p
        برای هر خال یک جعبه در نظر می‌گیریم. بنابراین ۴ جعبه داریم. هر بار که یک کارت انتخاب می‌کنیم، آن را در جعبه‌ی خال خودش قرار می‌دهیم. اگر 
        $N$
        تعداد کارت‌هایی باشد که انتخاب کردیم، طبق اصل لانه‌ی کبوتری برای اینکه دست کم ۳ کارت هم‌خال شوند باید 
        $\left \lceil \frac{N}{4} \right \rceil \geq 3$
        برقرار باشد. بنابراین داریم: 
        \begin{center}
            $N = 2 \times 4 + 1$
        \end{center}
      }
  
      \item 
      چه تعداد از کارت‌ها را باید انتخاب کنیم تا خال دست کم ۳ تا از کارت‌ها دل باشد؟
  
      \problemsolution{
        \p
       برای حل این قسمت از لانه‌ی کبوتری استفاده نمی‌کنیم. ابتدا تمام خال‌های دیگر را با بیرون کشیدن ۳۹ کارت انتخاب می‌کنیم. پس فقط کارت‌های خال دل باقی می‌مانند که با بیرون کشیدن سه تا از آن‌ها مسئله حل می‌شود. یعنی باید ۴۲ کارت بیرون بکشیم تا مطمئن باشیم خال سه تا از آن‌ها دل است.
      }
    \end{enumerate}
\end{problem}