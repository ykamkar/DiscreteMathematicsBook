\SECTION{اصل لانه کبوتری}

\begin{definition}
    \focused{اصل لانه کبوتری:}
    اگر 
    $k+1$
    کبوتر  بخواهند در 
    $k$
    لانه قرار گیرند، دست کم دو کبوتر در یک لانه قرار خواهند گرفت.
\end{definition}

\begin{theorem}
    مطلب بالا را می‌توان به شکل کلی‌تر، تحت عنوان
    \focused{تعمیم اصل لانه کبوتری}
    اینگونه بیان کرد که
    اگر 
    $N$
    شیء را در 
    $k$
    جعبه قرار دهیم، آنگاه دست کم یکی از جعبه‌ها دارای
    $\left \lceil \frac{N}{k} \right \rceil$
    شیء خواهد بود.
\end{theorem}

برای نمونه،
بین ۳۶۷ نفر، حداقل دو نفر با ماه و روز یکسان در تاریخ تولد وجود دارد؛ زیرا تنها ۳۶۶ تاریخ تولد متمایز در یک سال وجود دارد.
همچنین اگر نمره‌های ممکن برای درس ساختمان گسسته 
$A$
، 
$B$
، 
$C$
، 
$D$
و
$E$
باشد، برای اینکه دست کم ۶ تا از دانشجویان نمره‌ی یکسان بگیرند، این درس باید حداقل ۲۶ دانشجو داشته باشد.

\subfile{example 1.tex}
\subfile{example 2.tex}
\subfile{example 3.tex}