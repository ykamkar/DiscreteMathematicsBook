	\p
۳۰ چکمه را در ۳ بلوک ۱۰تایی چکمه افراز می‌کنیم. طبق اصل لانه کبوتر، یکی از این ۳ بلوک هست که حداقل ۵ لنگه چپ دارد. اگر دقیقا ۵ لنگه چپ داشت، که مسئله حل می‌شود. پس می‌توان فرض کرد که این لیوک ۱۰تایی حداقل ۶ لنگه چپ دارد.

حال دو بلوک حداکثر ۹ لنگه چپ دارند. بنابراین یکی از این دو بلوک حداکثر ۴ لنگ  چپ دارد. حال بلوک‌های ۱۰تایی از کفش را در نظر گرفته و یکی یکی آن را شیفت می‌دهیم. فرض کنید لنگه کفش‌ها را از راست به چپ با
$S_{1}, S_{2}, S_{3}, ..., S_{30} $
نشان بدهیم.
        بلوک اول لنگه‌های 
        $S_{1}$ تا $S_{10}$
        را در بر می‌گیرد و بلوک بعدی 
        $S_{2}$ تا $S_{11}$ را. به همین ترتیب فرض کنید بلوک \lr{i} ام 
        $n_{i}$ تا لنگه چپ داشته باشد.
        می‌دانیم که 
$n_{i}$ با هر بار شیفت دادن، یا بی نغییر مانده یا یک واحد تغییر می‌کند.

نشان هم دادیم که 
$n_{i}$ در یک جا مقداری کمتر یا مساوی ۴ دارد و در جای دیگر مقداری بیشتر یا مساوی ۶. از آن‌جایی که 
$n_{i}$
یک واحد یک واحد تغییر می‌کند حتما اندیسی مثل $k$ داریم که 
$n_{k}$ مساوی با ۵ شود.  