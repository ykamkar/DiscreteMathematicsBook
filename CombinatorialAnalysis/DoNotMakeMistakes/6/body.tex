\DMMproblem
با استفاده از توابع مولد نشان دهید تعداد روش‌های انتخاب ۴ عضو دو به دو  نامتوالی از مجموعه اعداد 
$1,2,3,...,n$
برابر است با 
انتخاب $4$ از $n-3$.
  
\DMMproblemWsolution{
    یک زیرمجموعه از این نوع مثلا {۱و۳و۷و۱۰} را انتخاب و نابرابری‌های اکید 
    $$0 < 1 < 3 < 7 < 10 < n+1$$
    را در نظر می‌گیریم. و بررسی می‌کنیم چند عدد صحیح بین هر دو عدد متوالی از این اعداد وجود دارند. در اینجا ۰ و ۱ و۳ و
    \lr{2}
    و n-۱۰ 
    را به دست می‌آوریم: ۰ زیرا عددی صحیح بین
    \lr{0}
    و ۱ وجود ندارد و ۱ زیرا تنها عدد ۲ بین ۱ و۳ وجود دارد و
    \lr{3} 
    زیرا اعداد صحیح ۴ و۵ و۶ بین ۳ و۷ وجود دارند و... .
    مجموع این ۵ عدد صحیح برابر
    \lr{ $ 0 + 1 + 3 + 2 + n-10 = n-4 $} 
    است.
   
    پس تابع مولد زیر را داریم.
  
    $$G(x) = ( 1 + x^2 + x^3 +...)^2 (x + x^2 + x^3 +...)^3 $$
    $$= ( \sum_{k = 0}^{\infty} x^k)^2 ( \sum_{k = 0}^{\infty} x^{k+1})^3 $$
    $$= \frac{1}{(1-x)^2} \times (\frac{x}{1-x})^3 = \frac{x^3 }{(1-x)^5} $$
    $$= x^2 (1-x)^{-5} = x^3 \sum_{k=0}^{\infty} \binom{k+4}{k} x ^ k $$
    $$= \sum_{k=0}^{\infty} \binom{k+4}{k} x^{k+3} = \sum_{k=0}^{\infty}\binom{k+1}{k-3} x^k$$
    
    به دنبال ضریب
    $ x ^ {n-4 } $
    می‌گشتیم پس 
    $ k = n-4 $
    و جواب نهایی برابر است با
    \lr{$\binom{n-3}{n-7} = \binom{n-3}{4}$} 
}

\NOTE{مثال زدن باید به صورتی باشد که حذف آن اختلالی در فهم جواب ایجاد نکند. در اینجا اگر مثال پاراگراف اول را حذف کنیم مشخص نیست تابع مولد برچه اساسی نوشته شده است. پس باید توضیحی درمورد تابع مولد و جمله‌ای که به دنبال ضریب آن هستیم بدهیم.}
\NOTE{
  نیاز هست که کاملا گفته شود چه تغییر متغیری انجام می‌شود. در اینجا تغییر متغیر $ k \rightarrow {k+3} $   را داریم.  همیشه به هنگام تغییر متغیر توجه کنیم ممکن است کران‌ها تغییر کنند. در اینجا کران پایین از صفر به سه می‌رود. 
  صورت اصلاح شده:
  $$ \sum_{k=3}^{\infty}\binom{k+1}{k-3} x^k $$
}
\NOTE{در طی پاسخ به سوال خوب است دقت کنیم همه‌ی اعداد را یا فارسی یا انگلیسی بنویسیم.}

\DMMproblemWsolution{
    تابع مولد فاصله از مبدا:
    \begin{align*}
    G(x)=(1+x+x^2+...)(x+x^2+x^3+...)^3
    \end{align*}
    در مجموع n-4 عدد داریم. توان‌های x باید بین مبدا و مقصد باشند پس باید توانی از x را که کوچک تر یا مساوی n-4 هستند را بیابیم:

      $$G(x)= \frac{x^3}{(1-x)^4} = x^3 (1-x)^{-4} = x^3 \sum_{k=0}^{\infty}\binom{k+3}{3} x^k$$
      $$\longrightarrow \sum_{k=0}^{\infty}\binom{x+k}{k} = \frac{1}{(1+x)^{k+1}}$$
      $$\longrightarrow k+3 \le n-4 \rightarrow k \le n-7$$

    \begin{align*}
      \binom{n+1}{r+1} = \sum_{k=r}^{n} \binom{k}{r} \hspace{2cm} (1)
    \end{align*}

    مجموع حالات: 
    \begin{align*}
    \longrightarrow \sum_{k=0}^{n-7}\binom{k+3}{3} \xrightarrow{(1)}  \binom{n-7+4}{4} = \binom{n-3}{4}
    \end{align*}
}
 

\NOTE{
  به هنگام جایگذاری در فرمول باید جایگذاری‌ها واضح باشد. در این مثال در فرمول (۱) کران پایین از r هست ولی در قسمتی که از آن استفاده شده کران پایین از ۰ است. همین مطلب گویای آن است که به توضیحات بیشتری نیاز هست. 

  عبارت زیر صورت کامل شده این نکته است:
  $$\longrightarrow \sum_{k=0}^{n-7} \binom{k+3}{k} = \sum_{k=3}^{k-4} \binom{k}{k-3} = \sum_{k=3}^{n-4} \binom{k}{3}$$
  $$\xrightarrow[{r \rightarrow 3},{ n \rightarrow {n-4} }]{(1)} \binom{n-3}{4}$$
}

\DMMproblemsolution{
  تعداد عضو‌های  انتخاب نشده کوچکتر از عضو اول انتخاب شده را 
  $x_1$،
  عضوهای انتخاب نشده بین عضو اول و دوم انتخاب شده را
  $x_2$،
  عضو‌های انتخاب نشده بین عضو دوم و سوم انتخاب شده را
  $x_3$،
  عضو‌های انتخاب نشده بین عضو سوم و چهارم انتخاب شده را
  $x_4$ و
  عضوهای انتخاب نشده بزرگ‌تر از چهارمین عضو انتخاب شده را 
  $x_5$
  می‌گیریم. کافی است تعداد جواب‌های صحیح نامنفی معادله زیر را با شرایط 
  $$x_1, x_5 \geq 0 \; x_2 x_3, x_4 \geq 1$$ 
  $$x_1 + x_2 + x_3 + x_4 + x_5 = n - 4$$
  بشماریم
  که برابر است با ضریب
  $x^{n - 4}$
  در عبارت:
  

  $$(1 + x + x^2 +...)\times(x + x^2 + x^3 +...)\times(x + x^2 + x^3 +...)$$
  $$\times(x + x^2 + x^3 +...)\times(1 + x + x^2 +...) = \frac{x^3}{(1 - x)^5}$$
  
   بنابراین کافی است ضریب $x^{n - 7}$ را در بسط
    $(1 - x)^{-5}$
بشماریم. 
 
 
     طبق جدول  
    Functions Generating Useful
از کتاب Rosen
که استاد نیز به آن اشاره کردند داریم:


$$(1 - x) ^ {-n} = \sum_{k = 0}^{\infty} \binom{n + k - 1}{k} x^{k} $$

بنابراین در این سوال داریم:
  
      $$(1 - x)^{-5} = \sum_{k = 0}^{\infty} \binom{5 + k - 1}{k}x^k$$
      $$= \sum_{k = 0}^{\infty} \binom{k + 4}{k}x^k$$
      $$= \sum_{k = 0}^{\infty} \binom{k + 4}{4}x^k$$

  $x^{n - 7}$ به ازای $k = n -7$ ساخته می‌شود. بنابراین جواب برابر است با:
  
 
      $$\binom{n - 3}{4}$$
}