\DMMproblem
\p
اتحاد زیر را ثابت کنید.
  $$1^2 {n \choose 1} + 2^2 {n \choose 2} + 3^2 {n \choose 3} + ... + n^2 {n \choose n} = n(n+1)2^{n-2}$$
  
\DMMproblemWsolution{
    \p
    فرض کنید $ P = \displaystyle\sum_{k=0}^{n} {{k^2}\binom{n}{k}}$ بیانگر تعداد راه‌های انتخاب یک کمیته از بین $n$ کاندیدا است به طوری که یک فرد یا دو فرد متمایز، رئیس کمیته باشند. حال این شمارش را به روش دیگری انجام می‌دهیم.
    \p
    با فرض داشتن یک رئیس، رئیس را انتخاب کرده و تصمیم می‌گیریم که بقیه افراد حضور داشته باشند یا خیر و حالات به دست آمده را جمع می‌کنیم با حالاتی که 2 رئیس را انتخاب کردیم در مورد حضور یا عدم حضور بقیه افراد تصمیم گرفتیم:
        \[P = n\times{2^{n-1}} + n\times(n-1)\times{2^{n-2}} = n\times(n+1)\times{2^{n-2}}\]
    \p
    از تساوی این ۲ حالت حکم مسئله اثبات می‌شود:
        \[\displaystyle\sum_{k=0}^{n} {{k^2}\binom{n}{k}} = n\times(n+1)\times{2^{n-2}}\]
}
 
\NOTE{عدم تطابق توضیحات با فرمول نوشته شده, انتخاب دو رئیس از میان $n$ نفر $\binom{n}{2}$ حالت دارد نه $n\times(n-1)$.}
\NOTE{بهتر است روش اثبات (دوگانه شماری) ذکر شود.}
\NOTE{یک طرف دوگانه شماری که نیازمند اثبات است, بدیهی در نظر گرفته شده است.}

\DMMproblemsolution{
    \p
    سوال را با دوگانه شماری حل می‌کنیم:
    \p
    فرض کنید $ P$ بیانگر تعداد راه‌های انتخاب یک کمیته از بین $n$ کاندیدا است به طوری که یک فرد رئیس کمیته و یک نفر معاون باشند و رئیس و معاون می‌توانند یک نفر باشند. شمارش این راه‌ها به 2 روش امکان پذیر است.
    \begin{enumerate}
        \item 
        با فرض یکسان بودن رئیس و معاون، رئیس را انتخاب کرده و تصمیم می‌گیریم که بقیه افراد حضور داشته باشند یا خیر و حالات به دست آمده را جمع می‌کنیم با حالاتی که رئیس و معاون متمایز را انتخاب کردیم و در مورد حضور یا عدم حضور بقیه افراد تصمیم گرفتیم:
        \[P = n\times{2^{n-1}} + n\times(n-1)\times{2^{n-2}} = n\times(n+1)\times{2^{n-2}}\]
        \item
        ابتدا این که چه اعضایی کمیته و رئیس و معاون را تشکیل دهند را انتخاب می‌کنیم که این تعداد می‌تواند هر عددی باشد، سپس رئیس و معاون یکسان یا متمایز را از بین آن ها انتخاب می‌کنیم:
        \[P = \displaystyle\sum_{k=0}^{n} {\binom{n}{k}(k(k-1)+k)} = \displaystyle\sum_{k=0}^{n} {{k^2}\binom{n}{k}}\]
    \end{enumerate}
    \p
    از تساوی 2 حالت فوق حکم مسئله اثبات می‌شود:
    \[\displaystyle\sum_{k=0}^{n} {{k^2}\binom{n}{k}} = n\times(n+1)\times{2^{n-2}}\]
}