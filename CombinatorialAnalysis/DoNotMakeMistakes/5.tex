\DMMproblem
چند عدد طبیعی حداکثر ۹ رقمی وجود دارد که مجموع ارقام آن برابر با ۳۲ باشد؟

\DMMproblemWsolution{
    سوال را با اصل شمول و عدم شمول حل می‌کنیم: 
    
     \[|A_1\cup A_2\cup... \cup A_9|=\]
     \[\binom{9}{1}|A_1|+\binom{9}{2}|A_1\cap A_2|+...+\binom{9}{9}|A_1\cap A_2\cap...\cap A_9|\]\\
     حال مقدار عبارت‌ها را حساب می‌کنیم:
     
     \[|A_1|=\binom{30}{8}\]
     \[|A_1\cap A_2|=\binom{20}{8}\]
     \[|A_1\cap A_2\cap A_3|=\binom{10}{8}\]
     برای بقیه جمله‌ها جواب برابر ۰ است.
     
     حال از اصل متمم برای به دست آوردن جواب نهایی استفاده می‌کنیم:
     
        -کل حالات:
      \[\binom{40}{8}\]
        - حالات مطلوب:
      \[\binom{40}{8}- \binom{9}{1}\binom{30}{8}+\binom{9}{2}\binom{20}{8}-\binom{9}{3}\binom{10}{8}\]

    \begin{align*}
    \longrightarrow a_{12} = \binom{16}{12} 4 ^ {12}
    \end{align*}
}

\NOTE{تعریف متغیر‌های $A_i$ ضروری است, چون در غیر این صورت منظور از بقیه استدلال‌ها به هیج وجه مشخص نیست.}
\NOTE{اثبات و یا در صورت وضوح, اشاره به تقارن میان مجموعه‌ها برای استفاده از اصل شمول و عدم شمول به این شکل ضروری است.}

\DMMproblemsolution{
    رقم $i$ ام این عدد را با $x_i$ نشان می‌دهیم, بنابراین به دنبال یافتن تعداد جواب‌های صحیح معادله زیر هستیم:
    \[\sum\limits_1^9 x_i=32\]
    \[\forall i \in [1,9] : x_i\leq 9\]
  تعداد جواب‌های صحیح این معادله را به کمک اصل متمم پیدا می‌کنیم:
  
     -کل حالات: تعداد جواب‌های صحیح نامنفی معادله  $\sum\limits_1^9 x_i=32 $ .این یک معادله سیاله است و تعداد جواب‌های صحیح آن برابر است با:
     
     \[\binom{40}{8}\]
     
    -حالات نامطلوب: تعداد جواب‌های صحیح نامنفی معادله $\sum\limits_1^9 x_i=32$  
     به طوری که:
     $\exists i \in [1,9] : x_i\geq 10$
     
     حال اگر مجموعه حالت‌هایی که در آن $x_i\geq 10 $ است را با $A_i$نشان دهیم, کافی است تعداد اعضای اجتماع این مجموعه‌ها را بیابیم.
      
     طبق اصل شمول و عدم شمول و با توجه به تقارن میان $A_i$ ها داریم: 
     \[|A_1\cup A_2\cup... \cup A_9|=\]
     \[\binom{9}{1}|A_1|+\binom{9}{2}|A_1\cap A_2|+...+\binom{9}{9}|A_1\cap A_2\cap...\cap A_9|\]
   برای محاسبه مقدار عبارت‌ها, در معادله سیاله متناظر, در صورتی که $x_i\geq 10$ بود قرار می‌دهیم $x_i=y_i+10 $ و در غیر این صورت قرار میدهیم $x_i=y_i$, حال اگر تعداد $i$ هایی را که به ازای آن‌ها $x_i\geq 10 $ است را با $k$ نشان بدهیم, حال به دنبال تعداد جواب های صحیح نامنفی معادله سیاله $\sum\limits_1^9 y_i=32-10k $ هستیم, که برابر است با: 
   \[\binom{40-10k}{8}\] 
    حال مقدار عبارت‌ها را حساب می‌کنیم:
    
    \[|A_1|=\binom{30}{8}\hspace{2cm} (k=1)\]
    \[|A_1\cap A_2|=\binom{20}{8}\hspace{2cm} (k=2)\]
    \[|A_1\cap A_2\cap A_3|=\binom{10}{8}\hspace{2cm} (k=3)\]
    برای بقیه جمله‌ها جواب برابر ۰ است.
    
    پس کل حالات نامطلوب برابر است با:
    \[\binom{9}{1}\binom{30}{8}-\binom{9}{2}\binom{20}{8}+\binom{9}{3}\binom{10}{8}\]
     - حالات مطلوب طبق اصل متمم برابر است با:
     \[\binom{40}{8}- \binom{9}{1}\binom{30}{8}+\binom{9}{2}\binom{20}{8}-\binom{9}{3}\binom{10}{8}\]
}