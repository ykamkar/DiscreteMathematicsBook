\SUBSECTION{ترتیب و جایگشت}

\begin{definition}
    به هر روش قرار گرفتن چند شیء در کنار یک‌دیگر یک 
    \focused{جایگشت}
    از این اشیاء گفته می‌شود.
\end{definition}

\NOTE{در مسائل ترکیبیاتی معمولا «تعداد جایگشت‌ها» مدنظر است و گاهی از مواقع به اشتباه از واژه «جایگشت» بجای «تعداد جایگشت‌ها» استفاده می‌شود.}

\begin{definition}
    به هر روش قرار گرفتن چند شیء «به صورت خطی» (در یک صف) در کنار یک‌دیگر، یک
    \focused{جایگشت}
    \focused{خطی}
    از این اشیاء گفته می‌شود.
    معمولا منظور از جایگشت، جایگشت خطی است مگر آن که نوع متفاوت جایگشت ذکر شود.
\end{definition}

\begin{fact}
    طبق اصل ضرب (تعمیم‌یافته)، می‌توان نتیجه گرفت
    تعداد جایگشت‌های خطی $n$ شیء متمایز برابر $n!$ است؛
    چرا که فارغ از انتخاب‌های قبلی، زمان انتخاب عنصری که باید در جایگاه
    $i$ام
    قرار گیرد، پیشتر
    $i-1$
    عنصر انتخاب شده‌اند و برای انتخاب پیش رو،
    $n-i+1$
    حالت وجود دارد. بنابراین تعداد کل حالات انتخاب این دنباله برابر است با:
    $$\prod\limits_{i=1}^n (n-i+1) = \prod\limits_{i=1}^n i = n!$$
\end{fact}


\begin{definition}
    به هر روش قرار گرفتن
    $n$
    شیء دور یک دایره، یک 
    \focused{جایگشت}
    \focused{دوری}
    از این
    $n$
    شیء
    گفته می‌شود،
    اگر یک آرایش از دوران آرایش دیگری به دست آید، آن‌گاه این دو آرایش را هم‌ارز می‌دانیم.
\end{definition}

\NOTE{
    توجه شود که تغییر جهت چرخش به دور دایره موجب تمایز می‌شود. به عنوان مثال
    12345
    و
    34512
    نامتمایز اما
    12345
    و
    54321
    متمایز هستند.}

\begin{fact}
    طبق اصل تقارن، می‌توان نتیجه گرفت
    تعداد جایگشت‌های دوری $n$ شیء متمایز برابر $(n-1)!$ است؛
    چرا که اگر یک جایگاه از حلقه‌ی جایگشت دوری را نقطه شروع درنظر بگیریم،
    به تعداد
    $n!$
    جایگشت خطی خواهیم داشت.
    با توجه به اینکه هر 
    $n$تا
    از آن‌ها حاصل دوران یک جایگشت هستند
    (هر بار یکی از اعضا در نقطه شروع قرار می‌گیرد و همان دنباله تکرار می‌شود)،
    طبق اصل تقسیم، تعداد حالات متمایز برابر است با :
    $$\frac{n!}{n} = (n-1)!$$
\end{fact}

\begin{definition}
    هر انتخاب با ترتیب
    $r$
    عنصر از یک مجموعه
    $n$
    عضوی
    (یک جایگشت خطی بر زیرمجموعه‌ای 
    $r$
    عضوی از یک مجموعه
    $n$
    عضوی)،
    یک
    \focused{$r$-}
    \focused{ترتیب}
    از مجموعه
    $n$
    عضوی است.
    \focused{
    ترتیب 
    $r$
    از 
    $n$}
    به معنای تعداد
    $r$-ترتیب‌های ممکن
    از یک مجموعه $n$ عضوی
    بوده که آن را با 
    $P(n,r)$ نشان می‌دهیم. 
\end{definition}

\begin{fact}
    اگر $n$ و $r$ اعدادی حسابی باشند به قسمی که 
    $r\leq n$، داریم:
    $$P(n,r) = n(n-1)(n-2)...(n-r+1) = \frac{n!}{(n-r)!}$$
\end{fact}

\NOTE{یک جایگشت خطی از یک مجموعه $n$ عضوی، درواقع یک $n$-ترتیب از آن مجموعه است.}

\subfile{CombinatorialAnalysis/PermutationsandCombinations/Permutation/example1/body.tex}