\SUBSECTION{ترکیب}

\begin{definition}
    هر انتخاب بدون ترتیب
    $r$
    عنصر از یک مجموعه
    $n$
    عضوی
    (یک زیر مجموعه $r$ عضوی از یک مجموعه $n$ عضوی)،
    یک
    \focused{$r$-ترکیب}
    از مجموعه
    $n$
    عضوی است.
    \focused{ترکیب
    $r$
    از 
    $n$}
    به معنای تعداد
    $r$-ترکیب‌های ممکن
    از یک مجموعه $n$ عضوی
    بوده که آن را با نماد 
    $C(n,r)$ یا 
    ${n\choose r}$ نمایش می‌دهیم. 
\end{definition}

\begin{fact}
    اگر $n$ و $r$ اعدادی حسابی باشند به قسمی که 
    $r\leq n$، داریم:
    $${n \choose r} = \frac{n(n-1)...(n-r+1)}{r!} = \frac{n!}{r!(n-r)!}$$
\end{fact}

\NOTE{
    رایج است که بجای عبارت
    «ترکیب $r$ از $n$»
    از عبارت 
    «انتخاب $r$ از $n$»
    استفاده شود.
}

\NOTE{
    دیدیم که یک
    $r$-ترکیب
    از مجموعه $A$،
    معادل یک زیر مجموعه $r$ عضوی از این مجموعه است.
    همچنین یک
    $r$-ترتیب
    از مجموعه $A$،
    جایگشتی خطی بر یک زیر مجموعه $r$ عضوی از این مجموعه است.
    مشخص است که یک $r$-ترتیب 
    حاصل محاسبه یک جایگشت خطی بر روی یک $r$-ترکیب است.
    بنابراین، می‌توان ترتیب را با کمک ترکیب تعریف کرد و طبق اصل ضرب می‌توانیم بنویسیم :
    $$P(n,r) = C(n,r)\times P(r,r)$$
    که $P(r,r)$ (مطابق انتظار) تعداد جایگشت‌های خطی را محاسبه می‌کند.
}

\NOTE{
    واضح است که:
    $${n \choose r} = {n \choose n-r}$$
}

\begin{fact}
    \focused{ترکیب تعمیم‌یافته:}
    اگر داشته باشیم
    $$n_{1} + n_{2} + ... + n_{k} = n$$
    آنگاه تعداد حالات افراز یک مجموعه $n$ عضوی
    به $k$ زیر مجموعه‌ی نام‌دار (متمایز) به نحوی که
    اندازه مجموعه
    $i$ام
    برابر $n_{i}$
    باشد، طبق اصل ضرب برابر است با:
    $${n \choose n_{1}} {n-n_{1} \choose n_{2}} ... {n_{k} \choose n_{k}}$$
    $$= \frac{n!}{n_{1}!(n-n_{1})!} \times \frac{(n-n_{1})!}{n_{2}!((n-n_{1})!-n_{2})!} \times ... \times \frac{n_{k}!}{n_{k}!1!}$$
    $$= \frac{n!}{n_{1}!n_{2}! ... n_{k}!}$$
\end{fact}

\begin{definition}
    برای سادگی، ترکیب تعمیم‌یافته را به شکل زیر نیز نمایش می‌دهیم:
    $${n \choose n_{1}} {n-n_{1} \choose n_{2}} ... {n_{k} \choose n_{k}} = {n \choose n_{1},n_{2}, ... n_{k}}$$
    و می‌خوانیم
    «ترکیب $n_1$ و $n_2$ و ... و $n_k$ از $n$».
\end{definition}

\subfile{example 1/body.tex}
\subfile{example 2/body.tex}
\subfile{example 3/body.tex}