\begin{extra}{
    فاکتوریل مقادیر حقیقی\\
    و\\
    ترکیب مقادیر حقیقی از مقادیر حقیقی
}
شاید برایتان جالب باشد اگر بدانید عملگر فاکتوریل ($!$)
برای اعداد غیر حسابی نیز تعریف شده است.
برای تعریف این عملگر، می‌دانید:
$$x! = x \times (x-1)!$$
پس:
$$(x-1)! = \frac{x!}{x}$$

درواقع اینکه می‌گوییم
$0! = 1$
نیز از همین رابطه نشأت می‌گیرد:
$$0! = \frac{1!}{1!} = 1$$

توجه کنید که در تعریف این عملگر، محدودیتی برای عملوند
($x$ در نوشتار بالا)
لحاظ نشده است پس می‌توانیم از همین قوانین برای دیگر مقادیر نیز استفاده کنیم.
به بررسی مقادیر منفی می‌پردازیم:
$$(-1)! = \frac{0!}{0} = \frac{1}{0}$$
که یک مقدار تعریف نشده است. بنابراین مقدار $(-1)!$ در مجموعه اعداد حقیقی تعریف نمی‌شود.
با منطقی مشابه می‌توان نشان داد فاکتوریل هیچ یک از اعداد صحیح منفی نیز تعریف نمی‌شود.
به عنوان مثال، مقدار $-3$ را درنظر بگیرید:
$$(-3)! = \frac{(-2)!}{-2} = \frac{(-1)!}{-2\times-1} = \frac{0!}{-2\times-1\times 0} = \frac{1}{0}$$

دلیل تعریف نشدن فاکتوریل اعداد صحیح منفی، تقسیم به صفر بود. اما درباره مقادیر غیر صحیح چه می‌توانیم بگوییم؟
برای مقادیر غیر صحیح نیز قوانین بالا صادق است اما چون تقسیم بر صفر رخ نمی‌دهد، فاکتوریل این مقادیر قابل تعریف است.
برای تعریف فاکتوریل مقادیر غیر صحیح از تابعی به نام «تابع گاما» با تعریف زیر استفاده می‌شود
که چون خارج از مباحث ریاضیات گسسته است، به جزئیات آن وارد نمی‌شویم.
$$\Gamma(s) = \int_0^\infty t^{s-l} e^{-t} dt$$

اگر با قوانین انتگرال آشنا باشید، به سادگی می‌توانید تابع بالا را برای مقدار خاص
$-\frac{1}{2}$
محاسبه کنید:
$$\Gamma(\frac{1}{2}) = \sqrt{\pi}$$

مقادیر دیگر نیز از طریق رابطه‌ی مشابه قابل محاسبه‌اند (بجز مقادیر صحیح منفی که تعریف نشده‌اند). دقت کنید که برای محاسبه‌ی 
فاکتوریل اعداد گویا، تنها نیاز است مقدار تابع بالا را در بازه‌ای به طول $[0,1)$ بدانیم، چرا که دیگر مقادیر از طریق آن‌ها قابل محاسبه‌اند.
مثال‌های زیر را درنظر بگیرید:

مثال) مقدار عبارات داده شده را بر حسب
$\pi$
بیابید.
\begin{enumerate}
    \item 
    $\frac{1}{2}!$

    \solution{
        $$\frac{1}{2}! = \frac{1}{2} \times (\frac{1}{2}-1)! = \frac{1}{2} \times -\frac{1}{2}! = \frac{1}{2} \pi$$
    }
    \item 
    $\frac{7}{2}!$
    
    \solution{
        $$\frac{7}{2}! = \frac{7}{2} \times \frac{5}{2} \times \frac{3}{2} \times \frac{1}{2}! = \frac{7 \times 5 \times 3}{2^3} \times \frac{1}{2} \pi$$
    }
    \item 
    $-\frac{5}{2}!$
    
    \solution{
        $$-\frac{5}{2}! = \frac{-\frac{1}{2}!}{-\frac{5}{2} \times -\frac{3}{2}}  = \frac{4\pi}{15}$$
    }
\end{enumerate}

حال که امکان محاسبه فاکتوریل مقادیر حقیقی را پیدا کردیم، می‌توانیم ترکیب مقادیر
حقیقی از مقادیر حقیقی را نیز محاسبه کنیم.

مثال)
مقدار عبارت زیر را برحسب $\pi$ محاسبه کنید:
$$3 \choose -\frac{5}{2}$$

\solution{
    $${3 \choose -\frac{5}{2}} = \frac{3!}{-\frac{5}{2}!\times \frac{11}{2}!} = \frac{6}{\frac{4\pi}{15} \times \frac{10395\pi}{64}} = 0.1385\pi^{-2}$$
}

اما همچنان درک مفهوم اعمال بالا بسیار دشوار است.
اگر اطلاعات بیشتری در این زمینه دارید، ممنون خواهیم بود اگر ما را در تکمیل این محتوا یاری کنید.
\end{extra}