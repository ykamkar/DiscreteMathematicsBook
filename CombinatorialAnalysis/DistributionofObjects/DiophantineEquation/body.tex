\SUBSECTION{معادله سیاله}

\begin{definition}
    \p
    \focused{معادله سیاله}
    \footnote{Diophantine Equation}
    در ریاضیات،
    معادله‌ای چند جمله‌ای با متغیرهای صحیح
    (مجهولات فقط می‌توانند مقادیر صحیح اتخاذ کنند)
    است. شکل کلی این معادلات را می‌توان به شکل زیر نمایش داد
    که در آن، تنها $x_i$ مجهول است
    (ضرایب و توان‌ها می‌توانند هر مقداری داشته باشند): 
    $$\sum\limits_{i}^{} {a_i} {x_i}^{p_i} = s$$
    \p
    \focused{معادله سیاله خطی}
    \footnote{Linear Diophantine Equation}
    معادله سیاله‌ای است که در آن، توان تمام مجهولات واحد هستند:
    $$\sum\limits_{i}^{} {a_i} {x_i} = s$$
    \p
    \focused{معادله سیاله خطی با ضرایب واحد}
    یک معادله سیاله خطی است که در آن، ضرایب تمام مجهولات برابر یک هستند:
    $$\sum\limits_{i}^{} {x_i} = s$$
\end{definition}

\p
این معادلات معمولا دارای چند پاسخ هستند.
به عنوان مثال معادله زیر را درنظر بگیرید:
$$x + y = 2$$
این یک معادله سیاله ساده است.
واضح است که هر زوج زوج
$(a, 2-a)$
می‌تواند یک جواب معادله برای زوج
$(x,y)$
باشد. بنابراین تعداد پاسخ‌های این معادله بینهایت است.
بسیار پیش می‌آید که هدف ما یافتن پاسخ‌های معادله در مجموعه اعداد طبیعی یا حسابی باشد.
این شرایط زمانی پیش می‌آید که یک مسئله طبیعی ساده را با معادله سیاله مدل کنیم
(مثلا می‌خواهیم از انواع مهره‌ها، تعداد متفاوتی برداریم به طوری که در نهایت ۱۰ مهره داشته باشیم).
در این صورت تعداد جواب‌های معادله بسیار محدود‌تر خواهد شد.
همان مثال بالا را اگر درنظر بگیرید؛ این معادله در مجموعه اعداد حسابی تنها ۳ پاسخ خواهد داشت.

\subfile{./DiophantineEquationwithConstantCoefficients/body.tex}
\subfile{./DiophantineEquationwithConstantCoefficients/example.tex}

\subfile{./ComplexDiophantineEquation/body.tex}