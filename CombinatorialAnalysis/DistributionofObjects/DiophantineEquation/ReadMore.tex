\begin{extra}{معادلات سیاله}
    \p
    \focused{معادله سیاله}
    در ریاضیات،
    معادله‌ای چند جمله‌ای با متغیرهای صحیح
    (مجهولات فقط می‌توانند مقادیر صحیح اتخاذ کنند)
    است. شکل کلی این معادلات را می‌توان به شکل زیر نمایش داد
    که در آن، تنها ‌$x_i$ها مجهول هستند
    (ضرایب و توان‌ها می‌توانند هر مقداری حقیقی داشته باشند): 
    $$\sum\limits_{i}^{} {a_i} {x_i}^{p_i} = s$$

    \p
    \focused{معادله}
    \focused{سیاله}
    \focused{خطی}
    معادله سیاله‌ای است که در آن، توان تمام مجهولات واحد هستند:
    $$\sum\limits_{i}^{} {a_i} {x_i} = s$$

    \p
        در معادلات سیاله، معمولا یکی از سوالات زیر مطرح است:
        \begin{enumerate}
            \item 
            آیا این معادله (در بازه‌ای محدود یا نامحدود) پاسخ دارد؟
            \item 
            این معادله (در بازه‌ای محدود یا نامحدود) چند پاسخ دارد؟
            \item 
            آیا قادر به محاسبه‌ی تمام پاسخ‌های معادله به صورت تئوری هستیم؟
            \item 
            آیا قادر به محاسبه‌ی حداقل یکی از پاسخ‌های معادله به صورت تئوری هستیم؟
            \item 
            آیا با استفاده از کامپیوتر، قادر به محاسبه‌ی تمام پاسخ‌های مسئله خواهیم بود؟
            \item 
            برای محاسبه‌ی تمام پاسخ‌های مسئله با استفاده از کامپیوتر، به چه میزان از منابع محاسباتی نیاز داریم؟
        \end{enumerate}

    \p
    همانطور که پیشتر مشاهده کردید، پاسخ به سوالات بالا برای معادلات سیاله خطی با ضرایب واحد، با استفاده از
    روش‌های توزیع اشیا، بسیار ساده بود.
    همچنین در ادامه مباحث این کتاب، در فصل نظریه اعداد، با روش کلی حل معادلات سیاله خطی
    آشنا خواهید شد.
    بررسی معادلات سیاله‌ی غیرخطی پیچیده‌تر می‌باشد و در پوشش این کتاب قرار نخواهد گرفت.
\end{extra}