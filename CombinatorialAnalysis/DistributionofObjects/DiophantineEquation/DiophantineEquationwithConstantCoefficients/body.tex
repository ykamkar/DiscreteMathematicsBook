\begin{problem}[معادله سیاله با ضرایب واحد در مجموعه اعداد حسابی]
    معادله سیاله زیر چند جواب دارد
    $$\sum\limits_{i=1}^n x_i = X$$
    اگر
    $x_i \in \mathbb{N}$
    و:

    \begin{enumerate}
        \item 
        $x_i \geq 0; 1 \leq i \leq n$
        \problemsolution{
            مسئله را به تقسیم
            $X$
            مهره نامتمایز
            به
            $n$
            جعبه متمایز مدل می‌کنیم.
            ادعا می‌کنیم مسئله معادل چینش با ترتیب
            $X$
            مهره و
            $n-1$
            مداد است.
            می‌دانیم
            $n-1$
            مداد به صف شده،
            $n$
            فضای متمایز تشکیل می‌دهند
            (بین هر دو مداد و قبل از اولین مداد و بعد از آخرین مداد).
            بنابراین می‌توانیم هر یک از این فضاها را به یک جعبه نسبت دهیم
            و مهره‌های قرار گرفته در هر فضا را درون جعبه‌ی متناظر قرار دهیم.
            پس این دو مسئله یکسان هستند.
            طبق جایگشت با اعضای تکراری، پاسخ این مسئله برابر است با:
            $$\frac{(X+(n-1))!}{X!(n-1)!}$$

            \begin{fact}
                به صورت کلی، پاسخ معادله سیاله با ضرایب واحد در مجموعه اعداد حسابی،
                با $n$ متغیر و مقدار ثابت $X$ برابر است با:
                $$\frac{(X+(n-1))!}{X!(n-1)!} = {X+n-1 \choose X}$$
            \end{fact}
        }
        \NOTE{
            به عبارت بالا دقت کنید. حاصل مسئله برابر است با ترکیب
            $X$
            از
            $X+n-1$.
            می‌توان این مشاهده را این گونه نیز تعبیر کرد که لازم داریم
            $X$
            مهره و
            $n-1$
            مداد ذکر شده در مدلسازی پاسخ را در
            $X+n-1$
            جایگاه متمایز، جایگذاری کنیم
            (بیانی دیگر برای به صف کردن).
            ابتدا، از بین این
            $X+n-1$
            جایگاه،
            $X$
            جایگاه را برای مهره‌ها انتخاب کرده و مهره‌ها را درون آن‌ها قرار می‌دهیم.
            سپس مداد‌ها را در جایگاه‌های باقیمانده قرار می‌دهیم و ادامه مدلسازی را مانند آنچه در بالا آمده
            دنبال می‌کنیم.
            بنابراین، تعداد حالات انجام این کار برابر
            ${X+n-1 \choose X}$
            خواهد بود.
            این ارتباط بین «ترکیب» و «جایگشت خطی با اعضای تکراری» را در بخش 
            «جایگشت خطی با اعضای تکراری»
            نیز به صورت کلی‌تر مشاهده کردیم.
        }

        \item 
        $x_i \geq t_i; 1 \leq i \leq n$
        \problemsolution{
            تغییر متغیر زیر را درنظر بگیرید:
            $$y_i = x_i - t_i; 1 \leq i \leq n$$
            بنابراین می‌توانیم معادله صورت سوال را به شکل زیر بازنویسی کنیم‌:
            $$\sum\limits_{i=1}^n x_i = \sum\limits_{i=1}^n (y_i + t_i) = X$$
            $$=> \sum\limits_{i=1}^n y_i = X - \sum\limits_{i=1}^n t_i$$
            برای درک بهتر مفهوم تغییر متغیر بالا، مسئله را با یک شبیه‌سازی تعبیر می‌کنیم.
            شرایط ذکر شده در این مسئله مانند آن است که در مسئله مهره و جعبه که در قسمت قبل ذکر شد،
            برای هر جعبه، تعداد حداقلی تعیین شود که حتما به آن تعداد مهره در آن جعبه قرار گیرد.
            راهکار حل این مسئله آن است که در ابتدا به تعداد خواسته شده مهره، درون آن جعبه‌ها قرار گیرد
            و سپس با مهره‌های باقیمانده، مسئله به یک مسئله معادله سیاله مدل شود.
            به این صورت تعداد
            $\sum\limits_{i=1}^n t_i$
            مهره از 
            $X$
            مهره‌ای که داشتیم کم شده و با تعداد باقیمانده مسئله را ادامه می‌دهیم.
            
            حال مطابق با نتیجه قسمت اول، پاسخ مسئله برابر است با:
            $$\frac{(X - \sum\limits_{i=1}^n t_i)+(n-1))!}{(X - \sum\limits_{i=1}^n t_i)!(n-1)!}$$
        }
    \end{enumerate}
\end{problem}