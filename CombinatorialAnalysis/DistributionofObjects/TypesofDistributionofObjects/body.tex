\begin{problem}[حالات مختلف توزیع اشیا]
  \begin{enumerate}
      \item 
        به چند طریق می‌توان ۷ توپ نامتمایز را در ۴ جعبه نامتمایز بدون محدودیت در ظرفیت آن‌ها، قرار داد؟
        \NOTE{این مسئله ترتیب با اشیا و جعبه‌های نامتمایز است.}

        \problemsolution{
          \p
          به ۹ روش می‌توان اینکار را انجام داد:
          \begin{center}
              $$6$$
              $$5,1$$
              $$4,2$$
              $$4,1,1$$
              $$3,3$$
              $$3,2,1$$
              $$3,1,1,1$$
              $$2,2,2$$
              $$2,2,1,1$$
          \end{center}
        }

      \item 
        اگر ۱۰ توپ که شماره‌گذاری شده‌اند را بخواهیم در ۳ جعبه شماره‌گذاری شده که هر کدام ۲ تا توپ ظرفیت داشته باشد، قرار دهیم، به چند حالت می‌توان این کار را انجام داد؟
        \NOTE{این مساله ترتیب با اشیا و جعبه‌های متمایز است.}

        \problemsolution{
            \p
            جعبه‌ی چهارمی برای توپ‌های باقیمانده درنظر می‌گیریم. حال طبق تعمیم ترکیب، تعداد روش‌های
            این تقسیم‌بندی برابر است با:
            $${10 \choose 2,2,2,4} = 420$$
        }

      \item 
        به چند روش می‌توان ۱۰ توپ نامتمایز را در ۸ جعبه شماره‌گذاری شده قرار داد؟
        \NOTE{این مساله ترتیب با اشیا نامتمایز و جعبه‌های متمایز است. این مسئله را با نام معادله سیاله نیز می‌شناسند.}
        
        \problemsolution{
          \p
          فرض کنید می‌خواهیم ۱۰ توپ و ۷ مداد را به ترتیب بچینیم و
          درنهایت، توپ‌های میان هر دو مداد، قبل از اولین مداد و یا بعد از آخرین مداد، هر کدام را درون یک جعبه قرار دهیم.
          ترتیب جعبه‌ها با ترتیب فضای بین مداد‌ها هم‌ارز خواهد بود.
          بنابراین لازم است تعداد جایگشت‌های ۱۰ توپ نامتمایز و ۷ مداد نامتمایز در کنار هم را بشماریم.
          طبق آنچه در جایگشت با اعضای تکراری بیان شد، این تعداد برابر است با:
          $$\frac{17!}{10!7!} = 19448$$
          \p
          روش بالا را تحت عنوان یک روش کلی برای حل معادله سیاله در بخش مربوطه خواهید دید.
        }

      \item 
        به چند طریق می‌توان ۴ توپ شماره‌گذاری شده را در ۳ جعبه نامتمایز بدون داشتن محدودیت در ظرفیت جعبه‌ها، قرار داد؟
        \NOTE{این مساله ترتیب با اشیا متمایز و جعبه‌های نامتمایز است. پاسخ این مسئله تحت عنوان عدد دوم استرلینگ نیز شناخته می‌شود.}
        
        \problemsolution{
          \p
          مسئله را بر روی تعداد جعبه‌های خالی حالت‌بندی می‌کنیم:
          \begin{enumerate}
              \item 
              اگر هیچ جعبه‌ای خالی نباشد، لازم است توپ‌ها را به یک دسته دو تایی و دو دسته‌ی تکی تقسیم کنیم. طبق تعمیم ترکیب، تعداد راه‌های انجام این کار برابر
              $${4 \choose 2,1,1} / 2! = 6$$
              است. توجه کنید که جعبه‌ها نامتمایزاند پس چون دو جعبه با تعداد اعضای مساوی داریم، لازم است حاصل ترکیب را بر تعداد جایگشت‌های آن‌ها تقسیم کنیم.

              \item 
              اگر دقیقا یک جعبه خالی بماند، یا توپ‌ها به دو دسته دوتایی تقسیم می‌شوند، یا به یک دسته سه‌تایی و یک دسته تکی.
              طبق تعمیم ترکیب و اصل جمع، تعداد راه‌های انجام این کار برابر است با:
              $${4 \choose 3,1} + {4 \choose 2,2} / 2! = 7$$

              \item 
              اگر دقیقا دو جعبه خالی باشد، یعنی ناچاریم همه توپ‌ها را درون جعبه آخر قرار دهیم:
              $${4 \choose 4} = 1$$
          \end{enumerate}
          \p
          طبق اصل جمع، پاسخ مسئله برابر است با:
          $$6 + 7 + 1 = 14$$
        }
  \end{enumerate}
\end{problem}