\begin{problem}[تعداد پریش‌ها]
  \p
    تعداد پریش‌های یک مجموعه
    $n$
    عضوی را بدست آورید.

    \problemsolution{
      \p
        عملگر
        $P(i)$
        را تعداد جایگشت‌هایی از مجموعه موردنظر تعریف می‌کنیم
        به نحوی که در این جایگشت‌ها، حداقل
        $i$
        عضو در جایگاه اصلی خود قرار داشته باشند.
        اگر پاسخ مسئله (تعداد پریش‌ها) را
        $N$
        و تعداد کل جایگشت‌های غیر پریش در این مجموعه را
        $A$
        بنامیم، طبق اصل متمم داریم :
        \begin{center}
          $N = P(0) - A$
        \end{center}
        همچنین طبق اصل شمول و عدم شمول داریم :
        \begin{center}
          $A = P(1) - P(2) + P(3) - ... + (-1)^{n-1} P(n) = \sum\limits_{i=1}^n (-1)^{i-1} P(i)$
    
          $\rightarrow N = \sum\limits_{i=0}^n (-1)^i P(i)$
        \end{center}
        \p
        برای بدست آوردن مقدار
        $P(i)$
        ابتدا 
        $i$
        عضو از مجموعه را انتخاب کرده و آن‌ها را در جایگاه اصلی خود قرار می‌دهیم.
        حال می‌توانیم به مسئله به چشم تعداد جایگشت‌های
        $n - i$
        عضوی باقیمانده نگاه کرد
        (جایگاه‌های اشغال شده و اشیا قرار داده شده را نادیده می‌گیریم).
        طبق اصل ضرب، 
        $P(i)$
        برابر حاصل ضرب تعداد حالات انتخاب
        $i$
        عضو و تعداد جایگشت‌های مجموعه جدید می‌باشد :
        \begin{center}
          $P(i) = \binom{n}{i} (n-i)! = \frac{n!}{(n-i)! i!} (n-i)! = \frac{n!}{i!}$
        \end{center}
        \p
        بنابراین داریم :
        \begin{center}
          $N = \sum\limits_{i=0}^n (-1)^i P(i) = \sum\limits_{i=0}^n (-1)^i \frac{n!}{i!} = n! \sum\limits_{i=0}^n \frac{(-1)^i}{i!} $
        \end{center}

        \begin{theorem}
          \p
            تعداد پریش‌های یک مجموعه 
            $n$
            عضوی برابر است با:
            $$n! \sum\limits_{i=0}^n \frac{(-1)^i}{i!}$$
        \end{theorem}
    }
\end{problem}