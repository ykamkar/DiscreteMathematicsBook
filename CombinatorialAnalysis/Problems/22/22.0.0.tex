\p
برای حل این سوال از دوگانه‌شماری استفاده می‌کنیم.

\begin{enumerate}
    \item 
    سمت چپ: 

\p
یک گروه $n$ نفره به یک رستوران می‌روند. منوی این رستوران چهار نوع پیتزا دارد.  یک نفر انتخاب می‌شود تا غذای همه را حساب کند($n$ حالت). بقیه ی $n-1$ نفر چهار انتخاب دارند. می‌دانیم کسی که غذا را حساب می‌کند، جهت صرفه‌جویی در هزینه‌ها! برای خودش پیتزای سبزیجات سفارش خواهد داد.

\item 
سمت راست:

\p
ابتدا $k$ نفری که قرار است پیتزایی غیر سبزیجات سفارش بدهند را انتخاب می‌کنیم 
$\binom{n}{k}$.
هرکدام از این افراد سه انتخاب برای پیتزا دارند. در نهایت فردی که قرار است غذاها را حساب کند انتخاب می‌کنیم. از آنجایی که این فرد پیتزای سبزیجات سفارش خواهد داد، از بین $n-k$ نفر انتخاب می‌شود. با جمع طرف راست برای تمام مقادیر ممکن $k$ متوجه می‌شویم که با طرف چپ برابری می‌کند و حکم مسئله ثابت می‌شود.
\end{enumerate}
