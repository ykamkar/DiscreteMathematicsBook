\p
حروف را به شکلی می‌شماریم تا کلمه‌ای متقارن با
$11$
حرف موجود بسازیم. به طور مثال، به این صورت آغاز می‌کنیم:
$$\bigcirc\bigcirc\bigcirc\bigcirc\bigcirc M\bigcirc\bigcirc\bigcirc\bigcirc\bigcirc$$
$30$
تعداد حالات شمارش دو حرف
$I$
، دو حرف
$S$
و یک حرف
$P$
برای قرار دادن سمت چپ یا راست 
$M$
است. اگر رشته‌ی حروف
$IPSSI$
را انتخاب کنیم، داریم:
$$IPSSIM\bigcirc\bigcirc\bigcirc\bigcirc\bigcirc$$
برای متقارن شدن کلمه، باید عکس آن یعنی رشته‌ی حروف
$ISSPI$
را در سمت دیگر قرار دهیم:
$$IPSSIMISSPI$$
در حالت کلی نیاز به شمارش حالات سمت دیگر نداریم، چرا که کل کلمه زمانی که رشته‌ی حروف یک سمت انتخاب شده باشد فیکس شده است. بنابراین تعداد کل حالات برابر با تعداد انتخاب‌های یک سمت، یعنی
$30$
می‌باشد.