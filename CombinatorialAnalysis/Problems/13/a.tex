\begin{enumerate}
  \item 
        سه کتاب فیزیک را یک کتاب واحد در نظر می‌گیریم. 
        $(7+1)!$
        
        از طرفی سه کتاب فیزیک به 
        $3!$ حالت به آن کتاب واحد تبدیل می‌شوند. پس 
        $8!\times3!$ حالت داریم.

  \item
        حالت قرار گرفتن کتاب‌ها به صورت زیر می‌شود($M$‌ها نشان دهنده‌ی کتاب‌های ریاضی و مربع‌ها نشان دهنده خانه‌های مجاز برای کتاب فیزیک است):
        \begin{center}
        $M_{1}\square M_{2}\square M_{3}\square M_{4}\square M_{5}\square M_{6}\square M_{7}$
        \end{center}
        برای قرار دادن کتاب‌های ریاضی
        $7!$ 
        حالت داریم. برای قرار دادن کتاب‌های فیزیک نیز باید باید یک ترتیب ۳‌تایی از ۶ حالت ممکن به دست بیاوریم. طبق اصل ضرب تعداد کل حالت‌ها برابر می‌‌شود با:
        \begin{center}
            $7! \times p(6,3)$
        \end{center}

  \item 
        حالت قرار گرفتن کتاب‌ها به صورت زیر می‌شود(مربع‌ها نشان دهنده خانه‌های مجاز برای کتاب فیزیک است و حروف MS به معنی امکان بودن کتاب شیمی یا ریاضی است):
        \begin{center}
        $M_{1}\square MS_{1}\square MS_{2}\square MS_{3}\square MS_{4}\square MS_{5}\square MS_{6}\square MS_{7}\square M_{2}$
        \end{center}
        برای کتاب‌‌های ریاضی
        $7!$
        حالت داریم. برای کتاب‌های شیمی نیز ابتدا دو جایگاه پیدا می‌کنیم که به 
        ${7\choose 2}$
        حالت امکان‌پذیر است؛ سپس به
        $2!$
        حالت می‌توان آن‌ها را در دو جایگاه انتخاب شده قرار داد. برای کتاب‌های فیزیک نیز ابتدا ۳ جایگاه از ۶‌تا به 
        ${6\choose 3}$
        حالت پیدا کرده و سپس به 
        $3!$
        حالت، کتاب‌های فیزیک را در آن جایگاه‌ها قرار می‌دهیم. (برای قرار دادن کتاب‌های شیمی و فیزیک می‌توانستیم از ترتیب استفاده کنیم، همانطور که در پاسخ قسمت قبل برای کتاب‌های فیزیک استفاده کردیم.) جواب نهایی برابر می‌شود با:
        \begin{center}
        $7! \times {7\choose 2}\times 2!\times {6\choose 3}\times 3!$
        \end{center}
\end{enumerate}