\documentclass[12pt,onecolumn,a4paper]{article}
\usepackage{epsfig,graphicx,subfigure,amsthm,amsmath,enumitem}
\usepackage{color,xcolor}
\usepackage{xepersian}
\settextfont[Path={./fonts/}, BoldFont={XB NiloofarBd.ttf}, BoldItalicFont={XB NiloofarBdIt.ttf}, ItalicFont={XB NiloofarIt.ttf}]{XB Niloofar.ttf}

\begin{document}
ابتدا حالات تعداد مختلف ممکن کتاب‌ها در قفسه ها را می‌شماریم:
$$x_1+x_2+x_3+x_4=24;\: x_1\geq 1;\: x_2\geq 1;\: x_3\geq 1;\: x_4\geq 1$$
$$\rightarrow y_1=x_1+1;\: y_2=x_2+1;\: y_3=x_3+1;\: y_4=x_4+1$$
$$\rightarrow y_1+y_2+y_3+y_4=20; y_1\geq 0; y_2\geq 0; y_3\geq 0; y_4\geq 0$$
$$\rightarrow n_A = {20+4-1 \choose 4-1} = {23 \choose 3}$$
حال پس از مشخص شدن تعداد جایگاه‌ها در هر قفسه و با توجه به متمایز بودن کتاب‌ها برای هر حالت تعداد کتاب‌ها در قفسه‌ها به 
$24!$
حالت می‌توانیم کتاب‌ها را بچینیم.

پس تعداد کل حالات:
$$n = {23 \choose 3} \times 24!$$
می‌باشد.
\end{document}