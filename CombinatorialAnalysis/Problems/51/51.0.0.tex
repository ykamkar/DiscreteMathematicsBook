\documentclass[12pt,onecolumn,a4paper]{article}
\usepackage{epsfig,graphicx,subfigure,amsthm,amsmath,enumitem}
\usepackage{color,xcolor}
\usepackage{xepersian}
\settextfont[Path={./fonts/}, BoldFont={XB NiloofarBd.ttf}, BoldItalicFont={XB NiloofarBdIt.ttf}, ItalicFont={XB NiloofarIt.ttf}]{XB Niloofar.ttf}

\begin{document}
برای حل این سوال از دوگانه‌شماری استفاده می‌کنیم.

سمت چپ: 

یک گروه $n+m$ از افراد داریم که $n$ نفر کلاه قرمز و $m$ کلاه آبی دارند. می‌خواهیم یک زیر مجموعه $n$ نفره از این افراد انتخاب کنیم.


سمت راست:

ابتدا $k$ نفری که قرار است کلاه آبی داشته باشند را انتخاب می‌کنیم.($\binom{m}{k}$)
سپس $n-k$ نفر باقی‌مانده را از کلاه قرمزها انتخاب می‌کنیم.( توجه کنید که
$ \binom{n}{n-k} = \binom{n}{k}$)
با جمع طرف راست برای تمام مقادیر ممکن $k$ متوجه می‌شویم که با طرف چپ برابری می‌کند و حکم مسئله ثابت می‌شود.
\end{document}