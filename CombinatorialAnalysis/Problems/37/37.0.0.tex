\p
برای حل این سوال از دوگانه‌شماری استفاده می‌کنیم. فرض می‌کنیم 
$2n$
پیراشکی با طعم‌های متفاوت داریم.
$n$
تا از آن‌ها در جعبه اول و 
$n$
تای دیگر در جعبه دوم هستند. می‌خواهیم 
$2$
پیراشکی برای صبحانه انتخاب کنیم. به دو روش می‌توانیم این کار را انجام دهیم:
\begin{enumerate}
\item
$\binom{2n}{2}$
حالت برای انتخاب آن دو از کل جعبه ها وجود دارد.
\item
هر دو پیراشکی را از جعبه اول انتخاب میکنیم:
$$\binom{n}{2}$$
هر دو پیراشکی را از جعبه دوم انتخاب میکنیم:
$$\binom{n}{2}$$
یک پیراشکی را از جعبه اول و دیگری را از جعبه دوم انتخاب میکنیم:
$$\binom{n}{1}\binom{n}{1}$$

\end{enumerate}
با توجه به برابر بودن حاصل دو روش فوق:

$$\binom{2n}{2} = \binom{n}{2} + \binom{n}{2} + \binom{n}{1}\binom{n}{1}$$
$$\binom{2n}{2} = 2\binom{n}{2} + n^2$$