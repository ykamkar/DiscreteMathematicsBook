    \p
    ابتدا جدول را به صورت شطرنجی رنگ‌آمیزی می‌کنیم. در این صورت هیچ دو خانه‌ی سیاهی مجاور ضلعی نخواهند بود. حال اگر هر کدام از خانه‌های سیاه را سفید کنیم، همچنان هیچ دو خانه‌ی سیاهی مجاور ضلعی نخواهند بود. پس جدول را به حداقل $2^{50}$ حالت می‌توان رنگ‌آمیزی کرد. دقت کنید که تعداد حالات‌رنگ آمیزی بیشتر از این است چون برای شطرنجی کردن جدول دو حالت وجود دارد.

    \p
    حال جدول را به 50 مستطیل 1×2 افراز میکنیم. هر یک از این مستطیل‌ها را می‌توان به حداکثر 3 حالت رنگ‌آمیزی کرد (هر دو خانه نمی‌توانند سیاه باشند جون با هم مجاورند). پس جدول را به حداکثر $3^{50}$ حالت می‌توان رنگ‌آمیزی کرد. دقت کنید که تعداد حالات رنگ‌آمیزی کمتر از این است چون رنگ‌آمیزی مستطیل‌ها مستقل از هم نیست و تعداد حالات رنگ‌آمیزی یک مستطیل با توجه با مستطیل‌های اطرافش می‌تواند کمتر از 3 باشد.

    \p
    در نتیجه:
    $$2^{50} < \text{تعداد حالات رنگ‌آمیزی مناسب} < 3^{50}$$
