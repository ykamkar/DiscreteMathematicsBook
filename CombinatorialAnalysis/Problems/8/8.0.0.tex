    \p
	تعداد ساعتی که بهزاد از تبلت, گوشی و لپ‌تاپ برای درس و لپ‌تاپ و کنسول برای بازی استفاده می‌کند را به ترتیب $x_1$, $x_2$, $x_3$, $y_3$, $x_4$ می نامیم.
    با توجه به صورت مسئله داریم:
    $$x_1, x_2 \leq 10$$
    $$x_3 + y_3 \leq 8 , x_4 \leq 12$$
    
    \p
    و با توجه به تعداد ساعتی که بهزاد می‌خواهد بازی کند و درس بخواند داریم:
    $$x_1 + x_2 + x_3 = 21$$
    $$y_3 + x_4 = 15$$
    
    \p
    با کمی تامل می‌توان فهمید که حتی در حالت حداکثری $x_1$ و $x_2$ حداقل مقدار $x_3$ برابر با 1 است و به همین ترتیب در معادله دوم حداقل مقدار $y_3$ برابر 3 است که حداکثر $x_3$ را طبق $x_3 + y_3 \leq 8$ به مقدار 5 کاهش می‌دهد. با در نظر داشتن کران‌های جدید و با استفاده از اصل شمول و عدم شمول روی $x_3$ حالت‌بندی می‌کنیم:
    $$\sum\limits_{x_3 = 1}^5 \underbrace{{22 - x_3 \choose 1}{16 \choose 1}}_{\text{کل حالات}} - \underbrace{2{11 - x_3 \choose 1}{16 \choose 1}}_{x_1 > 10 \; \text{یا} \; x_2 > 10} - $$
    $$\underbrace{{22 - x_3 \choose 1}{7 + x_3 \choose 1}}_{y_3 > 8 - x_3} - \underbrace{{22 - x_3 \choose 1}{3 \choose 1}}_{x_4 > 12} + $$
    $$\underbrace{2{11 - x_3 \choose 1}{7 + x_3 \choose 1}}_{x_1 > 10, y_3 > 8 - x_3 \; \text{یا} \; x_2 > 10, y_3 > 8 - x_3} + \underbrace{2{11 - x_3 \choose 1}{3 \choose 1}}_{x_1 > 10, x_4 > 12 \; \text{یا} \; x_2 > 10, x_4 > 12}$$
    
    $$= \sum\limits_{x_3 = 1}^5 (22 - x_3)(16 - 7 - x_3 - 3) - 2(11 - x_3)(16 - 7 - x_3 - 3)$$
    
    $$= \sum\limits_{x_3 = 1}^5 (6 - x_3)(22 - x_3 - 2(11 - x_3)) = \sum\limits_{x_3 = 1}^5 (6 - x_3)(x_3) = 35$$
