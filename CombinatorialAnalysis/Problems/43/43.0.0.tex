\p
    می خواهیم از n نفر، تعدادی را برای گروه 1 و تعدادی را برای گرو 2 برداریم، که الزامی ندارد تمام افراد در گروه ها قرار بگیرند و همچنین یک فرد در 2 گروه نمی تواند قرار بگیرد، این شمارش حالات به 2 صورت اتفاق می افتد که از برابری این 2 تعداد حالات حکم مساله اثبات می شود.
    \begin{enumerate}
        \item 
        ابتدا k نفر از n را انتخاب کرده سپس تصمیم میگیریم که هرکدام در گروه 1 باشند یا در گروه 2 و این کار را برای k های 0 تا n باید انجام دهیم و جمع کنیم:
        \(\displaystyle\sum_{k=0}^{n} {\binom{n}{k}\times2^k}\)
        \item
        از همان ابتدا برای هر فرد بین عضو هیچ گروهی نبودن یا عضو گروه 1 بودن یا عضو گروه 2 بودن، 3 انتخاب وجود دارد:
        \(3^n\)
    \end{enumerate}