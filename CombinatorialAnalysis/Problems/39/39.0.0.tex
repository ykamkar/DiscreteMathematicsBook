	\p
با قرار دادن تعداد مناسبی صفر در سمت چپ اعدادی که کمتر از ۹ رقم دارند، می‌توان آن‌ها را به اعدادی با دقیقا ۹ رقم تبدیل کرد. اگر 
$\overline{1x_2...x_9}$
 عدد مورد نظر باشد، داریم: 
 $x_1 + x_2 + ... + x_9 = 32$
  و
  $0 \leq x_i \leq 9$. \\
			 اگر تعداد کل جواب‌های این معادله را بشماریم، در بعضی جواب‌ها به ازای مقادیر $i$ خواهیم داشت $x_i \geq 10$ که نامطلوب هستند و باید از کل حالات کم شوند. اما نکته‌ای که وجود دارد این است که تعداد چنین $x_i$هایی که مقدار بیشتر مساوی ۱۰ می‌گیرند 
			 حداکثر ۳ تا می‌تواند باشد (در غیر این صورت مجموع ارقام بیشتر مساوی ۴۰ می‌شود که خلاف فرض سوال است)\\ 
			 
			 فرض کنید دقیقا $k$ رقم داشته باشیم که مقدار بیشتر مساوی ۱۰ گرفته‌اند.
			 همچنین فرض کنید $r(k)$ تعداد جواب‌های معادله زیر به شرطی باشد که $k$تا از $x_i$ها مقدار بیشتر مساوی ۱۰ گرفته باشند:\\
			 \begin{center}
			     $x_1 + x_2 + ... + x_9 = 32$
			 \end{center}
			 
			 فرض کنید آن $k$ رقم را انتخاب کرده‌ایم.
			 برای این کار $\binom{9}{k}$
			 حالت داریم.
			 می دانیم تعداد جواب‌های چنین معادله‌ای برابر با تعداد جواب‌های معادله زیر است:\\
			 \begin{center}
			     $y_1 + y_2 + ... + y_9 = 32 - 10k$ 
			     $y_i \geq 0$
			 \end{center}
			 می‌دانیم تعداد جواب‌های این معادله برابر است با:\\
			 \begin{center}
			     $\binom{32 - 10k + 9 - 1}{9 - 1}$
			 \end{center}
			 بنابراین داریم:\\
			 \begin{center}
			     $r(k) = \binom{9}{k}\binom{32 - 10k + 9 - 1}{9 - 1}$
			 \end{center}
			 
		    طبق اصل شمول و عدم شمول، جواب برابر است با:\\
			\begin{center}
			
			$r(0) - r(1) + r(2) - r(3) = \binom{9}{0}\binom{40}{8} - \binom{9}{1}\binom{30}{8} + \binom{9}{2}\binom{20}{8} - \binom{9}{3}\binom{10}{8}$
			
			\end{center}