ابتدا 
$4$
مهره سبز، بنفش، نقره‌ای و آبی بدون اعمال هیچ شرط خاصی، درون یک حلقه می‌چینیم. تعداد حالات مربوط به آن 
$3!$
 می‌باشد.
	\p
 سپس از
$4$
  جای بین آن ها 
$3$
   جا را برای 
$3$
    مهره باقی مانده که نباید مجاور هم باشند انتخاب می‌کنیم.
    \p
     \(P(4,3)\)
      همچنین طبق اصل تقارن، به این علت که می‌شود آن را از 
$2$
    سو نگاه کرد یک تقسیم بر 
$2$
     انجام می‌شود.
    \[\frac 1 2 \times3!\times P(4,3)\]