\p
می دانیم کارمند اول دقیقا 6 ساعت و کارمند دوم دقیقا 9 ساعت وقت می گذارد.
اگر مقدار ساعتی که کارمند اول برای کار \lr{i} انجام می‌دهد را با $x_i$ و مقدار ساعتی که کارمند دوم برای انجام کار \lr{i}  انجام می‌دهد را با $y_i$  نشان می‌دهیم:
$$ x_1 \geq 2 \quad y_1 = 0 $$
$$ x_2 \geq 0 \quad y_2 = 2x_2 $$
$$ x_3=x_4=x_5=0 \quad y_3  + y_4 + y_5 \geq 0 $$
\p
با توجه به اینکه با حل معادله سیاله اول مقدار $y_2$به صورت یکتا مشخص می شود و اینکه مجموع ساعات کارمند اول 6 ساعت و کارمند دوم 9 ساعت است بنابراین:
$$x_1 +x_2 = 6 \quad (x_1 \geq 2 , x_2 \geq  0)     \quad (1)$$
$$y_3 + y_4 + y_5 = 9-y_2 \quad (y_2 = 2x_2 , y_3 + y_4 + y_5  \geq 0) \quad (2)$$
\p
با توجه به ذات مسئله، می‌دانیم که مجموع ساعات کاری نمی‌تواند مقداری منفی داشته باشد و می‌توانیم معادله دوم را با روش معادله سیاله حل کنیم. برای این هدف (و برآورده کردن شرایط ذاتی مسئله) لازم است اطمینان حاصل کنیم که مقدار سمت راست معادله‌ی دو کوچکتر از صفر نمی‌شود. با توجه به اینکه
$x_1 + x_2 = 6$
و
$x_1 \geq 2$
پس
$x_2 = 6 - x_1 \leq 4$
و
$y_2 = 2x_2 \leq 8$
پس
$9 - y_2 \geq 1$
بوده و شرایط مذکور برقرار است.

$$ (1) x_1 = 2\rightarrow x_2=4 \rightarrow y_2=8 \rightarrow y_3+y_4+y_5 = 1 \rightarrow \binom{3}{2} = 3$$
$$ (2) x_1 = 3\rightarrow x_2=3 \rightarrow y_2=6 \rightarrow y_3+y_4+y_5 = 3 \rightarrow \binom{5}{2} = 10$$
$$ (3) x_1 = 4\rightarrow x_2=2 \rightarrow y_2= 4\rightarrow y_3+y_4+y_5 = 5 \rightarrow \binom{7}{2} = 21$$
$$ (4) x_1 = 5\rightarrow x_2=1 \rightarrow y_2=2 \rightarrow y_3+y_4+y_5 = 7 \rightarrow \binom{9}{2} = 36 $$
$$ (5) x_1 = 6\rightarrow x_2=0 \rightarrow y_2=0 \rightarrow y_3+y_4+y_5 = 9 \rightarrow \binom{11}{2} = 55 $$
\p
مجموع تعداد حالات: 
$$\sum_{k=1}^{5} \binom{2k+1}{2} = 125$$
