\begin{extra}{ضرایب دوجمله‌ای (چندجمله‌ای) تعمیم‌یافته}
\p
همانطور که در بیشتر بدانید «ترکیب مقادیر حقیقی» عنوان شد، این ترکیبات نیز قابل محاسبه‌اند.
جالب و کاربردی است بدانید که هرچند ما رابطه‌ی ضرایب چندجمله‌ای را با فرض صحیح
بودن توان عبارت اثبات کردیم، اما این قضیه برای توان‌های حقیقی نیز به همین شکل برقرار است
و می‌توان با محاسبه ترکیب از مقادیر حقیقی، بسط توان‌های حقیقی چندجمله‌ای‌ها را نیز بدست آورد.
تعمیم پذیری قضیه ضرایب چندجمله‌ای به توان‌های حقیقی اثبات پذیر است اما اثبات آن در حوزه ریاضیات گسسته قرار نمی‌گیرد.
با توجه به کاربردی بودن این قضیه که آن را تحت عنوان
\focused{ضرایب}
\focused{دو جمله‌ای (چندجمله‌ای)}
\focused{تعمیم}
\focused{یافته}
نیز می‌شناسند، یک مثال از آن را در زیر مشاهده می‌کنید.
\p
مثال)
ضریب عبارت
$x^4$
را در بسط
$(1+x)^{-3}$
بیابید.
\solution{
    $$(1+x)^{-n} = \sum\limits_{k=0}^\infty {-n \choose k} x^k = \sum\limits_{k=0}^\infty (-1)^k {n+k-1 \choose k} x^k$$
    بنابراین، پاسخ مسئله برابر است با:
    $$(-1)^4 {3+4-1 \choose 4} = {6 \choose 4} = 15$$
}
\end{extra}