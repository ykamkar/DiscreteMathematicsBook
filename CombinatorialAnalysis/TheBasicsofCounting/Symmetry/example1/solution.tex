\p
اگر از تساوی برخی از حالات با چرخش ۱۸۰ درجه صرف‌نظر کنیم و مهره‌های قلعه را متمایز در نظر بگیریم،
طبق اصل ضرب، تعداد چینش‌های ممکن برابر است با :
  $$T_1 = 64 \times 49 = 3136$$
که ۶۴ برای انتخاب یک خانه برای رخ اول و ۴۹ برای انتخاب یک خانه خارج از سطر و ستون
مربوط به رخ اول، برای رخ دوم است. برای از بین بردن تمایز بین می‌توان طبق اصل تقسیم نوشت :
  $$T_2 = \frac{T_1}{2} = \frac{3136}{2} = 1568$$
  \p
برای از بین بردن تمایز بین حالاتی که با چرخش ۱۸۰ درجه به هم تبدیل می‌شوند، با استناد بر این‌که
هر چیدمانی که در آن جایگاه دو رخ نسبت به مرکز صفحه متقارن نباشد، توسط چرخش ۱۸۰ درجه به
یک چیدمان جدید در پاسخ ما تبدیل می‌شود، طبق اصل تقسیم، تعداد این چیدمان‌ها را به دو تقسیم می‌کنیم.
توجه کنید که چیدمان‌هایی که در آن‌ها جایگاه دو رخ نامتمایز نسبت به مرکز متقارن است،
طی چرخش ۱۸۰ درجه به خودشان تبدیل می‌شوند، پس از ابتدا یک بار شمرده شده‌اند. تعداد این چیدمان‌ها
برابر است با : 
  $$T_3 = \frac{64 \times 1}{2} = 32$$
که در آن ۶۴ برای جایگاه رخ اول، ۱ برای جایگاه رخ دوم که ملزم است در تقارن رخ اول باشد
و تقسیم بر دو برای از بین بردن تمایز بین رخ‌ها طبق اصل تقسیم است.
طبق اصل متمم، تعداد چیدمان‌هایی که در آن‌ها جایگاه دو رخ متقارن نیست برابر است با :
  $$T_4 = T_2 - T_3 = 1568 - 32 = 1536$$
  \p
حال طبق اصل تقسیم، تعداد چیدمان‌های نامتقارن متمایز در برابر چرخش ۱۸۰ درجه برابر است با :
  $$T_5 = \frac{T_4}{2} = 1536 / 2 = 768$$
  \p
طبق اصل جمع، تعداد چیدمان‌های متمایز در برابر چرخش ۱۸۰ درجه، اعم از متمایز و نامتمایز برابر است با :
  $$T_6 = T_5 + T_3 = 768 + 32 = 800$$