\begin{problem}
    اگر به تعداد کافی سکه‌ی ۱ و ۲ و ۵ دلاری، داشته باشیم، به چند طریق می‌توان یک کالای ۷ دلاری را از دستگاه فروش خرید؟ به طوری که ترتیب انداختن سکه‌ها در دستگاه مهم باشد.

    \problemsolution{
        اگر 
        $n$
        ، تعداد کل سکه‌هایی که داخل دستگاه انداخته‌ایم باشد، تعداد حالت‌های انداختن سکه تا رسیدن به ۷ دلار با ضریب
        $x^{7}$
           در بسط 
        $(x + x^{2} + x^{5})^{n}$
            برابر است. زیرا این بسط دارای 
            $n$
             عبارت 
             $(x + x^{2} + x^{5})$
              است که در هم ضرب شده‌اند و ضریب
              $x^{7}$
         برابر است با تعداد حالت‌هایی که حاصل جمع توان جمله‌های 
         $x$
          از این عبارت‌ها برابر ۷ شود.      
               
             حال با جمع کردن تعداد جواب‌ها در 
            $n$
            ‌های مختلف، تعداد کل حالت‌ها به دست می‌آید. به عبارت دیگر تعداد کل حالت‌‌ها با ضریب 
            $x^{7}$
             در عبارت زیر برابر می‌شود.
            \begin{center}
                $1 + (x + x^{2} + x^{5}) + (x + x^{2} + x^{5})^{2} + ...$
            \end{center}
        با توجه به اینکه در توان‌های کمتر از ۲ و بیشتر از ۷، جمله‌ی 
        $x^{7}$
         را نداریم، از بررسی و محاسبه‌ی آن‌ها در عبارت چند‌جمله‌ای بالا صرف نظر می‌کنیم. با محاسبه‌ی عبارت، ضریب 
        $x^{7}$
        برابر ۲۶ به دست می‌آید. 
    }
\end{problem}