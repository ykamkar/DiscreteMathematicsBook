\SECTION{شمارش به کمک توابع مولد}
مسائل شمارش را می‌توان با یک مدل‌سازی ساده، به کمک توابع مولد حل کرد.
این کار همیشه منجر به افزایش سهولت حل نیست اما در برخی از مسائل چاره گشاست.
نحوه این استفاده را با چند مثال نشان خواهیم داد.

\SUBSECTION{شمارش بدون ترتیب}

\subfile{CombinatorialAnalysis/CountingwithGeneratingFunctions/GeneratingFunctions/example1.tex}
\subfile{CombinatorialAnalysis/CountingwithGeneratingFunctions/GeneratingFunctions/example2.tex}
\subfile{CombinatorialAnalysis/CountingwithGeneratingFunctions/GeneratingFunctions/example3.tex}

\SUBSECTION{شمارش با ترتیب}
برای کمک گرفتن از توابع مولد در حل مسائل شمارشی که در آن‌ها ترتیب با اهمیت است،
ساده‌تر است اگر از تابع مولد نمایی استفاده شود.
برای درک چرایی این مسئله، به یاد آورید که توانستیم
«ترتیب»
را به کمک حاصل ضرب
«ترکیب»
در
«جایگشت»
تعریف و محاسبه کنیم.
همانطور که در بخش قبل دیدید، می‌توان مسائل شمارش بدون ترتیب
را که حوزه عملکردی
«ترکیب»
است، به تابع مولد مدل کرد.
همچنین دیدیم که
تعداد جایگشت‌های
$n$ عضو
برابر $n!$
است.
بنابراین بنظر می‌رسد اگر بتوانیم
$n!$
را در پاسخ تابع مولد ضرب کنیم، آنگاه مسئله را به شکل ترتیب‌دار حل کرده‌ایم.
توجه کنید که در حل مسائل شمارش با کمک توابع مولد، ضریب جملات است که برای ما اهمیت دارد. بنابراین اگر در جملات
دنباله‌، مقدار
$\frac{1}{n!}$
را ضرب کنیم، آنگاه برای حل مسئله‌ی یکسان، برای خنثی شدن این اثر، در ضرایب جملات عبارت
$n!$
ضرب شده و ما به مقصود خود خواهیم رسید.
حال به یاد آورید که در فصل مجموعه‌ها ذکر شد، تابع مولد نمایی دنباله‌ی
$a_n$
همان تابع مولد دنباله‌ی
$\frac{a_n}{n!}$
است.
همانطور که مشاهده می‌کنید، استفاده از تابع مولد نمایی بجای تابع مولد در حل مسائل شمارش، هم معنا با اعمال تمایز
جایگشت‌ها در حالات مسئله است و مسائل بدون ترتیب را به مسائل با ترتیب تبدیل می‌کند.

روش استفاده از این روش را با چند مثال نشان خواهیم داد.

\subfile{CombinatorialAnalysis/CountingwithGeneratingFunctions/ExponentialGeneratingFunctions/example1.tex}
\subfile{CombinatorialAnalysis/CountingwithGeneratingFunctions/ExponentialGeneratingFunctions/example2.tex}
\subfile{CombinatorialAnalysis/CountingwithGeneratingFunctions/ExponentialGeneratingFunctions/example3.tex}