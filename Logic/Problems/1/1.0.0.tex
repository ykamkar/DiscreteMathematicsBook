\p
الف)
 سوال‌ 
        A
        : آیا اصغر راستگو است؟

        سوال 
        B
        : آیا اصغر برادرت است؟

        پاسخ این دو سوال چهار حالت ایجاد می‌کند :
        \begin{enumerate}
            \item 
            پاسخ
            A و B
            هر دو «بله» باشد. در این صورت شخصی که از آن سوال پرسیده‌ایم می‌گوید دیگری راستگو است پس خودش دروغ‌گو بوده
            و خودش اصغر و دروغگو است. پس جواب سوال اول «شخص پاسخگو» و جواب سوال دوم «راستگو» خواهد بود.
            \item 
            پاسخ
            A و B
            هر دو «خیر» باشد. در این صورت شخصی پاسخگو می‌گوید خودش اصغر بوده و برادرش اکبر و راستگو است.
            پس خودش دروغگو خواهد بود پس اکبر بوده و دروغگو. پس جواب سوال اول «برادر شخص پاسخگو» و جواب سوال دوم «دروغگو» خواهد بود.
            \item 
            با توجه به اینکه در دو مورد بالا حالاتی که شخص پاسخگو اکبر یا اصغر باشد و دروغگو را پوشش دادیم،
            فقط حالاتی باقی می‌ماند که شخص پاسخگو فقط راستگو باشد. پس نقیض پاسخ سوال
            A
            پاسخ سوال دوم و پاسخ سوال 
            B
            پاسخ سوال اول خواهد بود.
        \end{enumerate}
\p
ب) با پرسیدن تنها یک سوال ما قادر به پاسخ‌دهی به این سوال نمی‌باشیم.  با پرسیدن یک سوال حداکثر می‌توانیم چهار حالت را به دو دسته تقسیم کنیم، پس با پرسیدن یک سوال نمی‌توانیم به جواب قطعی برسیم.
